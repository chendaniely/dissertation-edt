\documentclass[040-assessment.tex]{subfiles}

\begin{document}

Putting together learning materials for learners typically usually leads to some kind of
assessment as to whether or not the materials created are effective
\cite{ambrose2010learning, wilson2019teaching}.
However, the term ``effective'' is vague and ill-defined.
This leads to the creation of using learning objectives,
concrete tasks and goals learners are expected to meet at the end of teaching instruction.
The benefit of creating learning objectives is they are able to be measured and assessed.
These assessments can be used to gauge the efficacy of a lesson.

\subsection{Assessments}

    Assessments come in 2 main forms: formative and summative.
    Formative assessments are the exercises students do during the course of instruction
    in order to monitor student learning
    \cite{universityFormativeVsSummative, wilson2019teaching}.
    They can interweave within a single instructional period (e.g., clicker type questions),
    or between instructional periods (e.g., quizzes, homework assignments).
    The main goal of formative assessments is to keep the learner engaged with the learning materials,
    and for both the instructor and learner to gauge learning by identifying areas that have not been grasped by the
    learner or areas that need more review.
    Formative assessments are typically low-stakes and given out frequently so the instructor
    can get an accurate gauge of how well the learners are doing
    \cite{universityFormativeVsSummative, wilson2019teaching}.

    In contrast, summative assessments are given at the end of an instructional period
    to evaluate student learning
    \cite{universityFormativeVsSummative, wilson2019teaching}.
    These types of assessments are the ``summation'' of multiple topics and can be given
    during a course of instruction (e.g., midterm exam)
    or towards the end of a course of instruction (e.g., final exam, thesis paper).
    Summative assessments are typically high-stakes and
    are used to gauge whether or not learners met learning objectives or
    prerequisite knowledge for subsequent courses or lessons
    \cite{universityFormativeVsSummative, wilson2019teaching}.

\subsection{Learning Objectives}

\end{document}
