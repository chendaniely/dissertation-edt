\documentclass[040-assessment.tex]{subfiles}

\begin{document}

Two (2) separate workshops were run.
All data and access to the exercises were conducted through Qualtrics.
Participants consented at the start of the study and provided a unique user identifier
(Appendix \ref{app:participant-id}).
The identifier was used to randomize participants into one (1) of four (4) treatment arms.
All exercise questions were created using R \code{learnr} documents,
depending on the treatment arm, each exercise question was paired with or without
the \code{gradethis} auto code grader.

The workshop began with a pre-workshop exercise.
During the workshop, three (3) 3 topics were covered and a short exercise (i.e., formative assessment)
was presented at the end of each topic.
The exercises fall into one of the four treatment arms for how the exercise is presented.
The end of the main workshop content was followed up with a final summative assessment question.
The amount of time to access the exercise and submit solution code was collected
along with user identifier and code solution for all of the coding questions.
Time to completion and code solutions were analysed and graded with a rubric.

\subsection{Treatment Arms}

    The study created 4 treatment arms to look at exercise type and whether or not an auto-grader for
    real-time feedback helps with student learning:
    (1) blank exercise + no auto grader,
    (2) faded example + no auto grader,
    (3) blank exercise + auto grader, and
    (4) faded example + auto grader.

    Blank exercises only contained the programming question and a space for participants to type and execute R code.
    Faded exercises contained the same programming question,
    but the space participants would type and execute R code would be pre-populated with the solution with
    function calls and arguments blanked out with a ``\_\_''
    (e.g., if the solution was ``read\_csv(`mydata.csv')'',
    the faded example would look like ``\_\_(\_\_)'').

\subsection{Randomization}

    Block randomization was used to randomize participants.
    A randomization list was pre-generated with a
    seed of 42, 4 treatment groups, block size of 8, and no stratification factors
    \cite{sealedenvelopeltdCreateBlockedRandomisation2021}.
    The second workshop re-generated the randomization list,
    but allocation ratios were adjusted for the second workshop due to varying amounts of participation
    and attrition after initial randomization from the workshop sign-in from the first workshop.
    Group 3 had 0 responses from the workshop
    The second randomization doubled the weight for Group 3,
    effectively creating a 5th treatment group, and used a block size of 10 for randomization.
    Participants were asked to sign-in at the start of the workshop,
    and their unique identifier was used for randomization.

\subsection{Workshop Content}

    The workshop delivered was  the ``Tidy Data'' portion of the ds4biomed materials %todo add citation.
    It starts with the understanding that learners know about
    (1) spreadsheets,
    (2) loading data into R,
    (3) subsetting columns and rows of data,
    (4) calculating grouped aggregate summary statistics.
    The workshop for this study started off with a pre-workshop survey that served as an assessment of
    prerequisite knowledge.
    The content workshop remained exactly the same as previous workshops and studies
    that covered tidy data principles.
    The only major change was forcing time for the formative and summative assessment questions.
    Participants were given 5 minutes to work on the pre-workshop and formative assessment questions
    and the solution was reviewed before continuing to the next topic.
    The summative assessment question solution was provided after the workshop to fit within the
    90-minute time limit for the workshop.

\subsection{Exercise Questions}

There was a total of 5 exercises presented to participants for this study:
(1) one pre-workshop exercise that asks participants to perform a small data pipeline task that they would
    have been taught by now in the full 6-hour workshop version,
(2) three formative exercise questions, and
(3) one final summative assessment question.
Each exercise question had a space for the participant to enter and run the existing R code.

\subsubsection{Pre-workshop Exercise}

    The pre-workshop exercise question is used as a baseline example since participants should
    be able to accomplish these tasks by this point of the workshop.
    The summative assessment example also uses concepts here during its data processing example.
    Participants were asked the following question:

    \begin{displayquote}

        Please write the code for the following pipeline steps:

        \begin{enumerate}
            \item Load the \code{tidyverse} and \code{readxl} libraries.
            \item Read in the Excel file located in:
                  \code{"data/medicaldata\_tumorgrowth.xlsx"} into a variable \code{tumor}.
            \item Select the all the columns except \code{Grp},
                  and filter the rows such that \code{Day} is \code{0} or \code{20}.
                  Save this data subset into a variable \code{tumor\_subset}.
            \item We want to compare baseline tumor sizes (Day 0) with tumor sizes at Day 20 between each of the groups.
                  Using \code{tumor\_subset},
                  calculate the average tumor \code{Size} for each \code{Grp} and \code{Day}.
            \item Save \code{tumor\_subset} into a CSV file located in \code{"data/tumorsubset.csv"}.
        \end{enumerate}

    \end{displayquote}

\subsubsection{Formative Assessment Exercise 1}

    \begin{displayquote}
        Take a look at the \code{ebola} dataset.
        %TODO HEAD EBOLA
        \begin{enumerate}
            \item Tidy the dataset such that you get the dataset below.
            \item You can use the \code{last\_col()} to select the last column of the dataset.
            \item Remember to drop missing values as the last step.
        \end{enumerate}

    \end{displayquote}

\subsubsection{Formative Assessment Exercise 2}
    \begin{displayquote}

        This is a different version of the \code{ebola} dataset.

        \begin{enumerate}
            \item Tidy the dataset such that you get the dataset below
            \item Remember to drop missing values as the last step
        \end{enumerate}

    \end{displayquote}

\subsubsection{Formative Assessment Exercise 3}

    \begin{displayquote}

        This is a different version of the \code{ebola} dataset.

        \begin{enumerate}
            \item Tidy the dataset such that you get the dataset below
            \item Remember to drop missing values as the last step
        \end{enumerate}

    \end{displayquote}

\subsubsection{Summative Assessment}

    The summative assessment question is the same question from the post-workshop and long-term survey question
    given to participants who participated in the full workshop sessions.
    The question given in the previous studies did not ask participants to code up the solution,
    rather it asked how confident participants are in their ability to complete a set of data tasks.
    This study asks the participants to actually code up the results that will be graded.

    \begin{displayquote}

        This is the cmv dataset you will load:

        \begin{enumerate}
            \item Use the \code{readxl} library to load the \code{"data/cmv.xlsx"} into a variable, \code{cmv}
            \item Filter the \code{cmv} dataset such that only $\text{age} > 65$ are remaining.
                  Save this to a variable, \code{cmv\_subset}.
            \item Save the \code{cmv\_subset} variable to a csv file in \code{"data/cmv\_subset.csv"}.
        \end{enumerate}

        \begin{enumerate}
            \item Tidy the \code{cmv} dataset such that it looks like the clean dataset below.
                  Save the tidy dataset into a variable, \code{cmv\_tidy}.
        \end{enumerate}

        \begin{enumerate}
            \item In the \code{cmv\_tidy} dataset, calculate the average age for each value of cmv.
        \end{enumerate}

    \end{displayquote}

\subsection{Grading Rubric}

    A grading rubric was created to score the participant-submitted code solutions.
    A composite score for each exercise solution was created based on multiple factors.
    1 point was awarded for each correct function call or similar function that achieves the same results.
    Grading was done with the actual treatment group blinded to the grader,
    scores were then combined with the rest of the full data for analysis.

    \paragraph{Pre-Workshop Exercise}

        7 points total.
        Submission was graded on:
        (1) loading library packages using the \code{library} function,
        (2) loading data using the \code{read\_excel} function,
        (3) subsetting columns using the \code{select} function,
        (4) subsetting rows based on a condition using the \code{filter} function,
        (5) aggregating summary statistics with the \code{group\_by} function,
        (6) calculating the mean on grouped variables with the \code{summarize} function, and
        (7) writing out results to an external file with the \code{read\_csv} function.


    \paragraph{Exercise 1}

        4 points total.
        Submission was graded on:
        (1) reshaping the dataset with the the \code{pivot\_longer} function,
        (2) correctly selecting the columns for the \code{pivot\_longer} function,
        (3) correctly specifying the new columns after the \code{pivot\_longer} function call, and
        (4) dropping missing values with \code{drop\_na} function.

    \paragraph{Exercise 2}

        6 points total.
        Submission was graded on:
        (1) reshaping the dataset with the the \code{pivot\_longer} function,
        (2) correctly selecting the columns for the \code{pivot\_longer} function,
        (3) splitting column values with the the \code{separate} function,
        (4) selecting the correct column to separate,
        (5) using the correct separating delimiter, and
        (6) ping missing values with \code{drop\_na} function.

    \paragraph{Exercise 3}

        3 points total.
        Submission was graded on:
        (1) reshaping the dataset with the the \code{pivot\_longer} function,
        (2) reshaping the dataset with the the \code{pivot\_wider} function, and
        (3) dropping missing values with \code{drop\_na} function.

    \paragraph{Summative Assessment Exercise}

        7 points total.
        Submission was graded on:
        (1) loading data using the \code{read\_excel} function,
        (2) subsetting rows based on a condition using the \code{filter} function,
        (3) saving the filtered dataset to a file with the \code{read\_csv} function,
        (4) reshaping and tidying the dataset with the the \code{pivot\_longer} function,
        (5) aggregating summary statistics with the \code{group\_by} function,
        (6) calculating the mean on grouped variables with the \code{summarize} function, and
        (7) writing out results to an external file with the \code{read\_csv} function.

\subsection{Analysis}

    Three (3) different sets of values were compared across each of the 4 treatment arms:
    (1) continuous variable of exercise solution score based on the grading rubric,
    (2) continuous variable of time to exercise completion, and
    (3) binary variable of whether the solution submitted is correct.
    The final results did not include the binary variable analysis due to sample size.

    Non-code solutions were dropped (e.g., pasted in a URL instead of solution code)
    and not graded, as opposed to a score of 0.
    Participants who attended both workshop sessions had their responses removed from the repeat session.
    Participants who attempted the exercise questions more than once had the higher score kept for analysis,
    in the event of a score tie, the first submission was used for analysis,
    and other responses were dropped.

\end{document}
