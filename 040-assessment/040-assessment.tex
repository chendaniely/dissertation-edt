\documentclass[../main.tex]{subfiles}
\begin{document}

\section*{Abstract}

A backward approach to lesson development start with identifying learner personas,
planning out what content needs to be covered and assessment questions.
These assessment questions can be used to outline the overall lesson
and used to guide leaners from one assessment question to another.
Desining assessment questions rely on the amount of information that is being taught
at any given point of a lesson.
Learner's are only able to keep about $4\pm1$ new bits of inforamtion in working memory.
This has rammifications about what kind of formative assessment questions
are asked during the teaching portion of a class.
Differnt types of exercise types can be used
to reduce cognitive load in an assessment question.

This study looks primiarly at how faded examples help with learner enguagement during a workshop
with and without an auto code grader.
Faded examples are questions that have the soluton partially removed (i.e., faded out),
and learners need to ``fill in the blanks'' to solve the solution.
This allows the cognitive load of the question to be reduced by filling in parts of the
solution that are not necessary for the conceptual point of the lesson.
Faded examples are then compared to regular assessment questions that only have the question,
and an empty space for the code solution.

The piolt study found that formative assessment questions are benefitial in the classroom,
regardless of the two exercise types used.
In the onlineline setting where the lesson was conducted,
a high percentage of responses were collected for the exercises compared to the expected amout of attrition.
This suggests that learners will enguage with formative assessment questions even when
there is little or no interaction during the online chat system.
Instructors are engouraged to provide ample time to complete these formative assessment questions
during active class intruction,
and provide additional learning resources for more details after the core concepts are taught.

\section{Introduction}

    \subfile{040-010-intro.tex}

\section{Methods}

    \subfile{040-020-methods.tex}

\section{Results}

    \subfile{040-030-results.tex}

\section{Discussion}

    \subfile{040-040-discussion.tex}

\end{document}
