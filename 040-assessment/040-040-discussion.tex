\documentclass[040-assessment.tex]{subfiles}

\begin{document}

\subsection{}

    practice testing instead of re-reading new matierials
    \cite{dunloskyImprovingStudentsLearning2013}.

\subsection{Formative assessments enguage students}

    very few questions or discussions occur in the zoom chatting platform.
    However, a surprisingly high number of students accomplised the exercises.
    The amount of attrition was less than expected.
    Especially when comparing it to attrition from registration to attendance.

    There were more participants who took the exercises during the workshop than
    questions and comments in the Zoom meeting room chat.

\subsection{Give Learners Time to Practice and Learn Asynchronously}

    Results from Figure \ref{fig:time-to-complete-combined-treatments} suggest more about
    how much more time it takes a novice learner to complete exercises compared to experts.
    Learners almost took the full 5-minutes for the formative assessment questions and
    almost the full 15-minutes for the summative assessment question.
    Using a conservative estimate for the instructor to go over the formative assessment solutions,
    learners take about 4-times as long to complete formative assessment questions,
    and almost 10-time as long to complete the summative assessment question.

    During 1-hour of instruction, this means about 15-minutes would be needed for formative assessments,
    leaving about 45-minutes to complete the main teaching materials.
    In a workshop setting over multiple hours or sessions, lessons can be balanced across other parts of the workshop.
    However, in individual workshop settings, additional time for setup would need to be
    considered for every lesson.

    This suggests that curating additional worked-out examples as formative assessment questions
    should to be provided to learners
    for asynchronous supplemental learning outside the main instructional period.

\end{document}
