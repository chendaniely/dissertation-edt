\documentclass[040-assessment.tex]{subfiles}

\begin{document}

    This was a pilot study looking at how different kinds of assessment questions
    and how they can be used to refine workshop content in a backward design approach.
    Some of the code submissions suggested that not all students utilized all parts of the
    coding platform,
    so the use of the auto-grader could not assumed it was used in treatment arms 3 and 4.
    The analysis was run with all 4 treatment groups, and with just 2 groups comparing
    the blank question with the faded question.
    Even with the low sample size from the study,
    there are still findings that are applicable to data science instructors.

\subsection{Formative Assessments Engage Students In Remote Workshops}

    The workshops were given in an online setting via Zoom.
    The observation from the instructor of the workshop was there was almost zero
    interaction of any kind during the workshop.
    The vast majority of chat messages came from the instructor posting the
    relevant links for each part of the workshop.
    Very few questions or discussions occur in the zoom chatting platform.
    There were more participants who took the exercises during the workshop than
    questions and comments in the Zoom meeting room chat.
    However, a surprisingly high number of students accomplished the exercises.
    The amount of attrition was less than expected,
    especially when comparing it to attrition from workshop registration to workshop attendance.

    This finding suggests that even without grades as an incentive,
    participants who volunteering opted in to participate in the workshop
    were engaged with the materials, even in an online zoom setting.

\subsection{Give Learners Time to Practice and Learn Asynchronously}

    Results from Figure \ref{fig:time-to-complete-combined-treatments} suggest more about
    how much more time it takes a novice learner to complete exercises compared to experts.
    Learners almost took the full 5-minutes for the formative assessment questions and
    almost the full 15-minutes for the summative assessment question.
    Using a conservative estimate for the instructor to go over the formative assessment solutions,
    learners take about 4-times as long to complete formative assessment questions,
    and almost 10-time as long to complete the summative assessment question.

    During 1-hour of instruction, this means about 15-minutes would be needed for formative assessments,
    leaving about 45-minutes to complete the main teaching materials.
    In a workshop setting over multiple hours or sessions, lessons can be balanced across other parts of the workshop.
    However, in individual workshop settings, additional time for setup would need to be
    considered for every lesson.

    This suggests that curating additional worked-out examples as formative assessment questions
    should to be provided to learners
    for asynchronous supplemental learning outside the main instructional period.

\end{document}
