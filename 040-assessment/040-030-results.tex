\documentclass[040-assessment.tex]{subfiles}

\begin{document}

The interpretations of the results from this study are limited due to attrition and low response rates.
The results are presented as preliminary data.
Due to the low sample size, treating the final summative assessment question as a binary variable of
correctness was not performed.
In general, none of the results had any statistical significance.

\subsection{Participants and Randomization}

    Two (2) workshop sessions that covered the same materials and exercise content were given.
    There were a total of 92 registrants (43 for session 1, and 49 for session 2)
    and 44 workshop attendees (25 for session 1, and 19 for session 2).
    30 participants were randomized across 4 arms
    (Group 1: 8, Group 2: 7, Group 3: 8, Group 4: 7).

    One (1) participant attended both workshops;
    Any data from the second workshop was dropped from analysis.
    One (1) participant attempted a question twice,
    since the graded scores were the same for both attempts,
    the first submission was kept for analysis.
    One (1) code solution appeared to be code for a different question and was not scored and counted in the analysis.
    A total of 16 randomized participants submitted at least 1 of the 5 exercises for analysis:
    1 participant only took 1 exercise,
    5 participants took 2 of the exercises,
    2 participants took 3 of the exercises,
    4 participants took 4 of the exercises, and
    4 participants took all 5 exercises.
    Table \ref{tab:exercise-treatment-response-counts} shows the number of
    responses for each exercise and treatment group.

    % from: 030-exercise_submission_descriptives.Rmd
    % latex table generated in R 4.1.1 by xtable 1.8-4 package
    % Thu Nov 25 23:34:34 2021
    \begin{table}[ht]
        \centering
        \caption[Number of responses by group and exercise]
        {Number of code submissions for each group and exercise.
         There were a total of 16 randomized participants submitted at least one (1) of the five (5) exercises for analysis.
         The values listed in the \code{Total} column are the exercise response counts,
         and the values listed in the \code{sum} column are the group counts.
         The columns represent the
         treatment group,
         exercise 1 (ex1), exercise 2 (ex2), exercise 3 (ex3),
         pre-workshop exercise (pre), and
         post-workshop summative assessment (sum).
        }
        \begin{tabular}{rlrrrrr}
            \hline
           & treatment & ex1 & ex2 & ex3 & pre & sum \\
            \hline
            1 & Group 1 &   1 &   3 &   3 &   4 &   4 \\
            2 & Group 2 &   3 &   4 &   2 &   3 &   2 \\
            3 & Group 3 &   1 &   1 &   1 &   2 &   1 \\
            4 & Group 4 &   5 &   4 &   3 &   3 &   3 \\
            5 & Total &  10 &  12 &   9 &  12 &  10 \\
             \hline
          \end{tabular}
          \label{tab:exercise-treatment-response-counts}
    \end{table}

\subsection{Exercise Scores}

    The pre-workshop exercise was given out as a means to cover pre-requisite knowledge for the
    summative assessment question.
    Otherwise, the summative assessment would be no different from any of the other exercise
    questions.
    Since the summative assessment exercise relied on knowledge that was covered in the
    pre-workshop exercise,
    the 5 participants who received a full score of 7 in the pre-workshop exercise were analysed first.
    Participants who scored well on the pre-workshop exercise also did well in all the other exercises
    (Figure \ref{fig:exercise-scores-pre100}).
    The 1 participant in Group 4 scored less than 50\% in the summative assessment question.
    Figure \ref{fig:exercise-scores-separate-treatements} shows the percentage scores across each
    of the exercises and tretment group.

    \begin{figure}[!htbp]
        \centering
        \includegraphics[width=0.9\textwidth]
        {figs/040-exercises/score\_prop-ex-treatment-facet_pre.png}
        \caption[Exercise scores of full scores and non full scores in pre-workshop exercise by treatment]
        {Distribution of scores for each treatment group across each exercise of
         respondents who received and did not recieve a full score in the pre-workshop exercise.
         Participants who scored well on the pre-workshop exercise also did well in all the other exercises.
         The 1 participant in Group 4 scored less than 50\% in the summative assessment question.
        }
        \label{fig:exercise-scores-pre100}
    \end{figure}

    \begin{figure}[!htbp]
        \centering
        \includegraphics[width=0.9\textwidth]{figs/040-exercises/score\_prop-ex-treatment-no\_facet.png.png}
        \caption[Graded Exercise Scores by treatment groups]
        {Distribution of exercise scores.
        Each participant's code submission was graded on a rubric.
        Scores are reported as a percentage because each exercise has a different total score.
        Score distributions were separate by whether or not a participant had a full score
        in the pre-workshop exercise (pre),
        since components of that are also used in the summative assessment question (sum).
        The other 3 exercises (ex1, ex2, ex3) were giving during the workshop
        as formative assessment questions.
        Results are also compared across 4 treatment groups:
        (1) Group 1: blank example with no auto code grader,
        (2) Group 2: faded example with no auto code grader,
        (3) Group 3: blank example with an auto code grader, and
        (4) Group 4: faded example with an auto code grader.
        Group 1 is the control group, and Group 4 is the main treatment of interest.

        The low sample size in each of the groups makes it difficult to make definitive conclusions.
        The data currently shows that participants who were able to get a full score
        in the pre-workshop exercise tend to also do well on the remaining exercise questions
        (Figure \ref{fig:exercise-scores-pre100}).
        More variation between exercises exists with participants who did not receive a full score
        in the pre-workshop exercise.
        }
        \label{fig:exercise-scores-separate-treatements}
    \end{figure}

    Some of the code solutions suggested that not every participant used the feature to execute their
    code befor submission.
    Since we were unable to confirm if the autograder was used in these exercise treatment groups,
    a separate analysis was performed that ignored the autograder,
    so treatment groups 1 and 3 were combined together and treatment groups 2 and 4 were combined together
    (Figure \ref{fig:exercise-scores-combined-treatments}).
    These preliminary results seem to suggest that faded examples do not affect code results during the formative
    assessments, but hinder results in the summative assessment question.
    These differences disappear when pre-workshop exercise performance is taken into account
    (Figure \ref{fig:exercise-scores-combined-groups-pre100}).

    \begin{figure}[!htbp]
        \centering
        \includegraphics[width=0.9\textwidth]{figs/040-exercises/score\_prop-ex-treatment-no\_facet-combine\_treatments.png}
        \caption[Graded Exercise Scores with combined treatment groups]
        {Distribution of exercise scores with the treatment groups combined by
        blank exercises (Groups 1 and 3) and faded examples (Groups 2 and 4).
        These groups were collapsed together since there was no way to track whether or not
        participants used the auto code grader.
        Preliminary results show that the faded examples do not differ from non-faded examples during
        the formative assessment questions,
        but the groups that used faded examples performed worse than those who were not given a faded example
        during the summative assessment question when all groups were provided an empty text field for the solution.
        }
        \label{fig:exercise-scores-combined-treatments}
    \end{figure}

    \begin{figure}[!htbp]
        \centering
        \includegraphics[width=0.9\textwidth]
        {figs/040-exercises/score\_prop-ex-treatment-facet\_pre-combine\_treatments.png}
        \caption[Exercise scores of full scores and non full scores in pre-workshop exercise by combined treatment]
        {Distribution of scores for combinded treatment groups across each exercise of
         respondents who received and did not recieve a full score in the pre-workshop exercise.
         Participants who scored well on the pre-workshop exercise also did well in all the other exercises.
         The 1 participant in Group 4 scored less than 50\% in the summative assessment question.
        }
        \label{fig:exercise-scores-combined-groups-pre100}
    \end{figure}

\subsection{Time to Complete}

    Next, we wanted to see if there were any differences between types of exercises and amount of time to complete an exercise.
    Faded examples provide a skeleton of the code for learners to fill-in instead of writing all of the code from scratch.
    There did not seem to be any difference between time to exercise completion between the non-faded groups with the
    faded example groups.
    However these results suggest more about the amount of time learners needed to work on exercises.

    \begin{figure}[!htbp]
        \centering
        \includegraphics[width=0.9\textwidth]{figs/040-exercises/time\_to\_complete-ex-treatment-no\_facet-combine\_treatments.png}
        \caption[Time to complete exercises with combined treatment groups]
        {Distribution of time to complete exercises between treatment groups combined by
        blank exercises (Groups 1 and 3) and faded examples (Groups 2 and 4).
        These groups were collapsed together since there was no way to track whether or not
        participants used the auto code grader.
        }
        \label{fig:time-to-complete-combined-treatments}
    \end{figure}

\end{document}
