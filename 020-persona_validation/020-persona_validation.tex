\documentclass[../main.tex]{subfiles}
\begin{document}

\section{Introduction}

\section{Methods}

    We created a pre-workshop student self-assessment survey (i.e., persona survey)
    where results were clustered to create and identify learner personas.
    Only complete survey responses were used for the analysis;
    there was no data imputation for missing or incomplete responses.
    The survey was validated with factor analysis and calculating Cronbach's alpha.
    Questions with high factor loadings were used in the clustering analysis for persona identification.
    The factor analysis results were not used to shorten the survey questions.
    Any duplicate response IDs were dropped.

    \subsection{Survey Creation}

        Survey questions were adapted from
        ``How Learning Works'',
        ``Teaching Tech Together'', and
        The Carpentries pre, post, and long-term workshop survey questions.
        We also added in additional questions, including demographic questions,
        for a total of 33 questions across 8 topics:
        (1) Demographics, 5
        (2) Programs used in the past, 1
        (3) Programming experience, 6
        (4) Data cleaning and processing experience, 4
        (5) Project and data management, 2
        (6) Statistics, 4
        (7) Data and programming likert table, 7.
        (8) Workshop framing and motivation, 3
        Two (2) of the workshop framing questions were free-response questions that asked about
        what learners hoped to learn in a workshop,
        and what they would they want to be able to do after the workshop they cannot do right now.

        Most questions were ordinal likert responses, however,
        expect for the actual likert table questions,
        we tried to make the choices more concrete,
        e.g., instead of asking to rate their abilities on completing a task as
        ``disagree'', ``neither'', or ``agree'', etc,
        we framed them as ``I wouldn't know where to start'',
        ``I could struggle through, but not confident I could do it'',
        ``I could struggle through by trial and error with a lot of web searches'',
        and
        ``I could do it quickly with little or no use of external help''.

        This survey is apart of a larger workshop longitudinal study.
        In order to protect privacy and not collect identifying information (e.g., names, email addresses, phone number, etc),
        we had users create a unique identifier that was generated based on their results to demographic questions.
        The IRB approved surveys can be found here: % TODO link this in bibliography https://github.com/chendaniely/dissertation-irb/tree/master/irb-20-537-data_science_workshops/survey

    \subsection{Survey Dissemination}

        Surveys were created using the Qualtrics system and emailed out in two (2) rounds.
        The first round was only sent out directly on biomedical relevant university listservs or to departmental administrators to post on our behalf at Virginia Tech.
        The second round included the same listservs and contacts from the first round,
        in addition to slack groups (The Carpentries, R/Medicine, Nursing \& Data Science Collaboration),
        emails collected from teaching 2 Carpentries instructor workshops one of them for the National Network of Libraries of Medicine (NNLM),
        and
        Claude Moore Health Science Library at the University of Virginia.

    \subsection{Survey Validation}

    % TODO add some text here, i don't like stacked headings
    % TODO maybe use the reference to the R packages her

        \subsubsection{Face Validity}

            We structured the survey around many data literacy concepts and asked question around
            programming, data, and statistics experience.
            We assumed people had some familiarity with spreadsheet programs (i.e., Excel),
            and focused the questions around Excel proficiency before asking questions
            around ``tidy data'' principles and more specific statistics analysis questions.
            These were all topics we were hoping to cover in preparing workshops materials,
            and the questions from the survey provided a place on where we would potentially start the workshop.
            These questions and their use case provided the face validity for the survey.

        \subsubsection{Factor Analysis}

            The ``psych'' R package was used for factor analysis.
            The demographic, free-response, and prior programming languages used questions
            were dropped and the rest of the responses were scaled prior to running the analysis.
            The scree plot from the R ``nFactors'' package suggested to try 1-factor to 5-factor models.
            We used varimax rotation with a maximum likelihood factoring method and
            the best model was picked based on the
            TLI, RMSEA, and BIC scores, in addition to model interoperability.

        \subsubsection{Internal Consistency}

            The ``psych'' R package was also used to calculate Cronbach's alpha for internal consistency.
            To ensure unidirectionality,
            a separate Cronbach's alpha measure was calculated for each set of questions that loaded into each factor.

    \subsection{Survey Clustering}

        Hierarchical clustering with euclidean distance and Ward's clustering method was used to identify respondent groups. % TODO cite R packages
        These groups were then combined with demographic information and survey responses to create the learner personas.

    \subsection{Learner Personas}

        Results from hierarchical clustering and survey responses was used to create the
        fictional learner persona characters.
        To make the character more complete,
        we combined the empirical results to write the
        background,
        relevant prior knowledge and experience,
        perception of needs,
        and special considerations
        for each persona.
        The background and special consideration sections were not backed by empirical data,
        but they were needed to create a complete persona.

\section{Results}

    % TODO calculate and fill in these numbers
    There were a total of ***** responses to the persona survey.
    **** were over 18 years old and also consented to being apart of the research study.
    **** complete observations were used for the factor analysis, Cronbach's alpha, and clustering analysis.
    of the ***** incomplete responses,
    **** completed ****\% of the survey.
    The persona survey had **** set of responses that shared a duplicate ID.
    Looking at the individual responses,
    It was determined that this particular set of IDs came from the same person,
    and only the initial set of responses were kept for analysis.
    Overall, we used the 3-factor model and the clustering results yielded 3 clusters for our personas.

    \subsection{Survey Descriptives}

    \begin{figure}
        % TODO this figure does not seem like it looked at only the results from validation + clustering.
        % TODO: i think the actual validation and clustering data only had upper 50 num of respondants
        % TODO: remake this figure without the N and num Responses subtitle values
        \centering
        %\includegraphics[width=0.7\linewidth]{survey/01-self\_assessment/grouped\_demographics.png}
        \caption[Grouped demographics for persona survey respondents]
            {Total number of survey respondents from the persona study and their self reported occupation.
                This particular survey question was a ``select all that apply''
                question with the ability to type in an ``other'' response.
                The responses were then grouped together into the 4 major occupation groups.
            }
        \label{fig:groupeddemographics}
    \end{figure}


\section{Discussion}

\end{document}
