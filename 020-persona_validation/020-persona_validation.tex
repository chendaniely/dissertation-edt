\documentclass[../main.tex]{subfiles}
\begin{document}

\section*{Abstract}

    % In this chapter/paper,
    % I confirm personas can be empirically created used in education
    % and demonstrate that they can also be empirically created for learners (students).
    % I demonstrate this process by creating and validating a survey where I then cluster responses together
    % The purpose of the work is to demonstrate that you are able to
    % create learner personas in the a specific domain (e.g., biomedical sciences) in order to
    % create more relevant and domain-specific data science learning materials.
    % this is written in a formal and analytic tone for educators wishing to create lesson materials

    % intro outline:
    % why make data science materials specific? people learn better.
    % are there reasons other domains want to use it? data science cuts across all fields
    % how would you go about making domain-specific materials? make learner personas
    % what are learner personas? they are fictional characters that represent a portion of your audience
    % how would you create them? i use a survey to cluster them
    % how do you know the survey is good? i also validate the surveys.
    % How does this help create lesson materials?

    Many data science learning resources are geared towards general audiences.
    This provides an opportunity to fill the need for more domain-specific learning materials
    which can provide more context, motivation, and reliability to the content to help engage learners.
    One of the first steps in creating new learning materials is identifying the audience who will
    be using the materials.
    Persona methodology provides a user-centered framework to empirically identify
    the learning audience (i.e., learner personas), how they will interact with the lesson,
    what their prior relevant knowledge are,
    and their perception of needs.
    This study demonstrates that personas can be empirically created used in education to create learner personas.
    Specifically, this study looks into learner personas in the biomedical sciences to
    create more relevant and domain-specific data science learning materials.

    A survey was distributed asking participants about their
    prior programming experience;
    data cleaning and processing experience;
    project management satisfaction; and
    statistics.
    The survey items were validated using Confirmatory Factor Analysis (CFA),
    and the learner personas were identified with hierarchical clustering.
    Survey responses were combined with the clusters to create 3 learner personas:
    Ash Academic, Clare Clinician, and Samir Student.
    The primary persona was Clare Clinician and was used to create a domain-specific
    set of learning materials for the biomedical sciences.
    The survey can also be used for future educators to identify their learners
    and can be used to create learner personas for other domains.


\section{Introduction}

\subfile{020-010-intro.tex}

\section{Methods}

    \subfile{020-020-methods.tex}

\section{Results}

    \subfile{020-030-results.tex}

\section{Discussion}

    \subfile{020-040-discussion.tex}

\section{Supplemental}

    Supplemental materials for the
    ``Identification of Biomedical Data Science Learner Persons:
             Implications and Lessons Learned for Domain-Specific Data Science Curriculum''.

    \subfile{020-050-010-survey_questions.tex}
    \subfile{020-050-020-results.tex}
    \subfile{020-050-030-learner_personas.tex}

\end{document}
