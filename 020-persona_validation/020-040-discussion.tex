\documentclass[020-persona\_validation.tex]{subfiles}

\begin{document}

    % TODO stacked heading

    There were a few existing parameters heading into this study.
    The target audience are:
    (1) adult working professionals,
    (2) work as a clinician and/or do research in the biomedical field, and
    (3) use Excel as the main, if not only, program for data work.
    These parameters already put in constraints into the lesson development phase:
    (1) The lesson needs to be modular and sections need to be taught within 1 hour blocks.
    This can account for busy work schedules and can possibly be given during break times (e.g., lunch).
    (2) Written lesson materials need to be provided online for asynchronous learning and can be used as reference material later on.
    (3) Classes will be recorded and posted online for the same reason written materials are posted.
    (4) Data examples must be medically or health related.
    (5) A separate lesson needs to be given about working with data in Excel and setting up spreadsheet data.
    (6) An example of loading an Excel spreadsheet into the programming language so learners can import any existing datasets.
    A separate study will build on these concepts to create concrete learning objectives (LOs) for the lesson materials.

    These studies are learner focused and does not focus on the skills needed to properly present the materials as an instructor.
    Teaching knowledge does play a factor in how materials are presented and how the instructor handles questions from the class.
    Assessing content knowledge, pedagogical content knowledge of the instructor was not performed
    \cite{shulmanThoseWhoUnderstand1986},
    but since all the workshops will be taught by the same instructor, those components were consistent throughout out
    the future workshops that assess the learning materials.
    The Carpentries Instructor Training materials is a good reference to learn more about evidence-based teaching practices
    \cite{Koch2016}.
    This study will provide a foundation for future instructional material creators identify core data literacy components
    that need to be taught.

    \subsection{Identification of Biomedical Data Science Learner Personas Informs Curriculum Design}

        There was a relatively even balance between the 3 main occupation groups: students, researchers, and clinicians.
        The student group mainly clustered into their own persona,
        which also happened to be the most experienced persona.
        The programming experience can be because higher education is beginning to incorporate data science courses and
        into existing programs,
        from students learning skills doing their own research.
        If the more experienced students come from a self-taught programming background,
        then a more formalized class can solidify their existing mental model,
        and build the way for more connections to progress their learning.

        Gender, race, and ethnicity demographic information was collected for these surveys.
        However, these results were not used for the persona clustering.
        Future iterations of this survey should not include these
        demographic questions since they are not relevant to creating the personas.

        The survey had 23 items with 57 responses for analysis.
        We performed a correlation analysis to reduce the number of items down to 14, for factor analysis
        but these values may change with more respondents.
        The main motivation to drop items from the factor analysis was due to the low number of responses.
        A general rule of thumb is to have 10 observations per item,
        and the correlation analysis to select items was one way to mitigate the effects of having a low number of observations.

        One of the assumptions for the ML factoring method is the items need to be normally distributed.
        Our analysis showed that our data was not normal,
        and many of the other factoring methods did not seem relevant for our data.
        The PA factoring method was selected because it does not assume the normality of data.
        This method better suited the data distribution and
        while it did not meet every cutoff value,
        we felt this was enough to show some validity for the survey,
        and can be used in a future study to collect more data.
        The Cronbah's alpha results gave us more confidence that our survey was adequately consistent and
        can be a useful took for researchers moving forward.

    \subsection{Creation of Biomedical Data Science Learner Personas}

        The demographic information was a main contributor on how the personas were named.
        It is possible that the names and number of personas will change with more responses.
        The benefits of creating these explicit learner personas are for future educators
        can have a more concrete understanding of their audience.
        These personas can be used to cater and focus learning materials and better serve learners' needs.
        The personas were primarily named by their occupation demographics.
        The overall differences between the personas are listed in Table \ref{tab:persona-summary-table}.
        In theory, combination of differences in Table \ref{tab:persona-summary-table} would be its own learner persona,
        but not every combination would require a different set of learner needs.

        \begin{table}[htb]
            \centering
            \caption[Persona Data Science Skill Rating]
            {How each persona differs from one another. In terms of data science and data literacy skills.
             Results come from persona survey results
            }
            \begin{tabular}{llll}
            \hline
            Group & Programming & Statistics & Data Programming \\%& Finding Technical Help \\
            \hline
            1     & Low         & Medium     & Medium           \\%&                        \\
            2     & High        & High       & High             \\%& High                   \\
            3     & Low         & Low        & Low              \\%&                        \\
            \hline
            \end{tabular}
            \label{tab:persona-summary-table}
        \end{table}

        \subsubsection{Samir Student (Group 2)}

            Samir Student (Group 2) is indicative of many introductory programming classes.
            They serve as the more experienced individuals who have previously seen the materials
            or have experience with data programming (Table \ref{tab:persona-summary-table}).
            These students may find some of the materials too simple for their needs,
            and can cause a distraction in the classroom by asking questions that are too complicated
            for the intended audience to understand.
            There are a few ways to manage these students in the classroom.
            Overly technical questions outside the scope of the lesson materials
            can be answered during a break or after the class as to not confuse the rest of the learners.
            These more experienced learners can also provide help to other learners around them.
            This will also reinforce their own understanding of the materials being covered,
            while keeping them engaged in the class.
            In a virtual setting,
            these students can provide help and answer questions in the text chat.

            This bimodal learner experience is also confirmed when comparing the
            2-cluster model with the 3-cluster model. % TODO put an image here
            Samir Student was also the first clustering split with survey responses showing
            they had the most experience with data science concepts.
            The 3-cluster model partitioned the less experienced programmers.

            Making the lessons modular and publicly available
            gives these types of learners the chance to judge if the materials are suited for their needs
            and experience levels.

        \subsubsection{Clare Clinician (Group 3)}

            Groups 1 and 3 are mainly ``never'' programmers.
            What distinguishes group 3 from the rest, is they are also not confident in
            their ability to perform a statistical analysis,
            and may not agree that programming makes analyses easier to reproduce (Table \ref{tab:persona-summary-table}).

        \subsubsection{Ash Academic (Group 1)}

            Group 1 was also mainly in the ``never'' programmers group.
            This group was also a catch-all for all the other occupations.
            Group 1 is more confident in their abilities to do statistical analysis.
            They also lean slightly toward programming languages being helpful for reproducible analysis.

    \subsection{More Data for Personas}

        The persona survey was given in 2 waves.
        The results in the first wave had 4 clusters: Expert, Clinician, Academic, and Student.
        The expert and student clusters combined into a single group with the addition of wave 2 data.
        We feel that as more data gets collected and the persona survey gets distributed to more
        instructors, we better stratify the respondents and more accurately identify learner personas and
        needs.

    \subsection{Domain-Specific Learner Personal Survey Validation}

        There are four (4) main benefits to using personas:
        (1) make assumptions about users explicit,
        (2) place the focus on specific types of users rather than on all possible users,
        (3) Help make better decisions by limiting choices
        (4) Engage the product design and development team
        \citet{pruittPersonaLifecycleKeeping2006, schwartzParadoxChoiceWhy2016}.
        As educators, these learner personas provide a starting point to curate, create, and design
        domain-specific learning materials for the biomedical sciences.
        The primary user-group of Clare Clinician can be used to focus on content and examples.
        The personas created are still in their initial conception phase,
        but they should have enough detail to begin planning lesson materials.
        Even if the personas are ``wrong'',
        the lesson materials will still be consistent for learners
        \citet{pruittPersonaLifecycleKeeping2006}.

\end{document}




    resuability pardox wile2003, tttgreag. better to remix materials than constantly re-use

    Need Two subsections in the Discussion if you have One. The second subsection should be Domain-Specific Learner Personal Survey Validation and talk about how this can be used in the future. A lot of the discussion needs to connect back to the problems the work is addressing and the impact it has, then the TLDR conclusions of what we really learned. that has to come out more clearly

"Case studies that show the explicit and implicit value of data reuse" \cite{payneBiomedicalInformaticsMeets2018}
''

The National Library of Medicine is one sugestion where biomedical infomratics and data science materials can be housed \cite{payneBiomedicalInformaticsMeets2018} after inital federal funding (e.g., NIH BD2K program) ends.
Beyond NIH,
\cite{payneBiomedicalInformaticsMeets2018}
NSF, PCORI, AHRQ, Foundations, Corporations, etc.
BMI: biomedical informatics; DS: data science; NLM: National Library of Medicine; NIH: National Institutes of Health; NSF: National Science Foundation; PCORI: Patient Centered Outcomes Research Institute; AHRQ: Agency for Healthcare Research and Quality.
