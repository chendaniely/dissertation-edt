\documentclass[020-persona\_validation.tex]{subfiles}

\begin{document}

    Results from a pre-workshop student self-assessment survey (i.e., persona survey)
    were clustered to create and identify biomedical data science learner personas.
    Only complete survey responses were used for the analysis.
    There was no data imputation for missing or incomplete responses.
    The learner persona survey was validated with factor analysis and calculating Cronbach's alpha.
    Clustering used all the questions in the survey,
    regardless of factor analysis results,
    for persona identification.
    The factor analysis results were not used to shorten the survey questions.
    All duplicate response IDs were identified as coming from the same person.
    These responses were coalesced and only the first set of responses were kept for analysis.Survey design, validation, and analysis are discussed below.
    The R programming language was used for all data processing and analysis
    \cite{allaireRmarkdownDynamicDocuments2021,
    auguieGridExtraMiscellaneousFunctions2017,
    bacheMagrittrForwardpipeOperator2020,
    bernaardsGradientProjectionAlgorithms2005, corrplot2021,
    csardiProgressTerminalProgress2019, dahlXtableExportTables2019,
    devriesGgdendroCreateDendrograms2020, epskampSemPlotPathDiagrams2019,
    firkeJanitorSimpleTools2021, gagolewskiStringiFastPortable2021,
    gagolewskiStringiFastPortable2021a,
    garnierViridisColorblindfriendlyColor2021,
    ginnQualtRicsDownloadQualtrics2021a, gohelFlextableFunctionsTabular2021,
    gohelOfficerManipulationMicrosoft2021, grolemundDatesTimesMade2011,
    henryPurrrFunctionalProgramming2020, henryRlangFunctionsBase2021,
    henryTidyselectSelectSet2021, hesterFsCrossplatformFile2021,
    hesterGlueInterpretedString2021, kassambaraFactoextraExtractVisualize2020,
    korkmazMVNPackageAssessing2014, maechlerClusterClusterAnalysis2021,
    mullerHereSimplerWay2020, mullerTibbleSimpleData2021,
    neuwirthRColorBrewerColorBrewerPalettes2014,
    oomsJsonlitePackagePractical2014, oomsWritexlExportData2021,
    perryUtf8UnicodeText2021, raicheNFactorsParallelAnalysis2020,
    rcoreteamLanguageEnvironmentStatistical2021,
    revellePsychProceduresPsychological2021, rosseelLavaanPackageStructural2012,
    therneauRpartRecursivePartitioning2019, wickhamDplyrGrammarData2021,
    wickhamForcatsToolsWorking2021, wickhamGgplot2ElegantGraphics2016,
    wickhamReadrReadRectangular2021, wickhamRvestEasilyHarvest2021,
    wickhamStringrSimpleConsistent2019, wickhamTidyrTidyMessy2021,
    wickhamWelcomeTidyverse2019a, xieDynamicDocumentsKnitr2015,
    xieKnitrComprehensiveTool2014, xieKnitrGeneralpurposePackage2021,
    xieMarkdownCookbook2020, xieMarkdownDefinitiveGuide2018,
    zhuKableExtraConstructComplex2021}

    \subsection{Learner Self-Assessment Survey (Persona Survey)}

        Survey questions were adapted from
        ``How Learning Works'',
        ``Teaching Tech Together'', and
        The Carpentries pre-, post-, and long-term workshop survey questions
        \cite{ambrose2010learning, wilson2019teaching, jordanAnalysisSoftwareData2018, jordanAnalysisCarpentriesLongTerm2020, jordanAnalysisCarpentriesLongTerm2018, jordanAnalysisCarpentriesLongTerm2017}.
        Additional demographic questions were also added.
        A total of 33 questions across eight (8) topics were included in the final survey:
        (1) Demographics, 5
        (2) Programs used in the past, 1
        (3) Programming experience, 6
        (4) Data cleaning and processing experience, 4
        (5) Project and data management, 2
        (6) Statistics, 4
        (7) Data and programming Likert table, 7
        (8) Workshop framing and motivation, 3.

        Two (2) of the workshop framing questions were used to design follow-up biomedical data workshops.
        These free-response questions asked:
        (1) what learners hoped to learn in a biomedical data science workshop, and
        (2) what they would they want to be able to do in working with data after a biomedical data science workshop
            or training event that they cannot do right now.

        \subsubsection{Survey Questions}

            Of the 33 questions in the survey, 16 were on an ordinal Likert scale.
            All survey questions included in the survey can be seen in
            (Supplemental \ref{sse:persona-survey-questions}).
            In order to better relate to learners and aid in their assessment of skills,
            Likert table questions were not asked in the more common
            ``Disagree'', ``Neither'', or ``Agree'', etc,
            format,
            but rather asked,
            ``I wouldn't know where to start'',
            ``I could struggle through, but not confident I could do it'',
            ``I could struggle through by trial and error with a lot of web searches'',
            and
            ``I could do it quickly with little or no use of external help''.
            These Likert responses were framed to aid in the clarity of selection of the survey taker,
            and potentially make the responses more consistent between participants.

            This survey is also a part of a larger workshop longitudinal study.
            In order to track participant responses longitudinally and protect their privacy by not collect identifying information
            (e.g., names, email addresses, phone numbers, etc),
            participants created a unique identifier that was generated based on their results to demographic questions.
            The IRB approved surveys can be found here in supplemental section \ref{sse:persona-survey-questions}.

        \subsubsection{Survey Dissemination}

            Surveys were created in the Qualtrics platform and emailed out in two (2) rounds
            \cite{Qualtrics2005}.
            The first round was only sent to biomedical relevant university listservs
            or to departmental administrators to post on our behalf at Virginia Tech.
            The second round included the same listservs and contacts from the first round,
            in addition to slack groups (The Carpentries, R/Medicine, Nursing \& Data Science Collaboration),
            emails collected from teaching two (2) Carpentries instructor workshops,
            one of them for the National Network of Libraries of Medicine (NNLM),
            and
            Claude Moore Health Science Library at the University of Virginia.

        \subsubsection{Survey Validation}

            % TODO maybe use the reference to the R packages her
            The learner self-assessment survey (i.e., persona survey)
            was created with the goal of becoming a tool for future instructors
            to help identify the learning audience.
            The only biomedical domain-specific questions involve the statistics related questions
            where the example uses a health-related research question.
            This was so participants had a better sense of what kind of analysis they would do given a particular scenario.
            All of the other questions do not assume any particular domain,
            and the statistics questions are framed in a way where domain knowledge is not needed.

            \paragraph{Face Validity}

                The survey was structured around many data literacy concepts and asked questions around
                programming, data processes, and statistics experience.
                It was assumed survey takers had some familiarity with spreadsheet programs (i.e., Excel),
                and focused the questions around spreadsheet proficiency before asking questions
                around ``tidy data'' principles and more specific statistics analysis questions.
                These were all topics we were hoping to cover in preparing workshops materials,
                and the questions from the survey provided a starting point for the workshop material content.
                These questions and their use case provided the face validity for the survey.

            \paragraph{Factor Analysis}

                The \code{psych} R package was used for factor analysis
                \cite{revellePsychProceduresPsychological2021}.
                The demographic, free-response, and prior programming languages used questions
                were not used and the rest of the responses were scaled prior to running the analysis.
                Items for factor analysis were picked based on how many other items they were significantly correlated with.
                The scree plot from the R \code{nFactors} package suggested trying 1-factor to 5-factor models
                \cite{raicheNFactorsParallelAnalysis2020}.
                Different rotations and factoring methods were tested,
                and the best model was picked based on
                TLI, RMSEA, and BIC scores, in addition to model interoperability.


            \paragraph{Internal Consistency}

                The \code{psych} R package was also used to calculate Cronbach's alpha for internal consistency.
                A separate Cronbach's alpha measure was calculated for each set of questions that loaded into each factor.
                Kendall's kappa was not calculated because survey results were not paired.

    \subsection{Identification of Biomedical-Specific Data Science Learner Personas}

        All survey questions were used for clustering;
        results from factor analysis did not select which questions would be used for clustering.
        Hierarchical clustering with Euclidean distance and
        Ward's clustering method was used to identify respondent groups
        \cite{rcoreteamLanguageEnvironmentStatistical2021}.
        These groups were then combined with demographic information and survey responses to create learner personas.

        Results from hierarchical clustering and survey responses were used to create the
        fictional learner persona characters.
        To make the character more complete,
        we combined the empirical survey results for the
        background,
        relevant prior knowledge and experience,
        perception of needs,
        and special considerations
        for each persona.
        The background and special consideration sections were not backed by empirical data,
        but they were needed to create a complete persona.

\end{document}
