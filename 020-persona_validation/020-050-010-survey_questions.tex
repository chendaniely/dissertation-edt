\documentclass[020-persona\_validation.tex]{subfiles}

\begin{document}

\subsection{Pre-Workshop Student Self-Assessment Survey (Persona Survey) Questions}

The surveys can be downloaded from the GitHub URL that holds the IRB proposal for the study:
\url{https://github.com/chendaniely/dissertation-irb/tree/master/irb-20-537-data\_science_workshops/survey}

    \subsubsection{Demographics}

        \paragraph{Q2.2}

            Please create a unique identifier.
            This unique identifier will be used for long-term assessment but keep your personal information anonymous.

            To create an identifier type in:
            Number of siblings (as numeric) +
            First two letters of the city you were born in (lowercase) +
            First three letters of your current street (lowercase).

            E.g., (Sherlock Homes has \textbf{1} brother,
                was born in \textbf{Po}rsmouth,
                and lives on \textbf{Bac}ker Street - \textbf{1pobac})

        \paragraph{Q2.5}

            What is your current occupation/career stage (select all that apply).

            \begin{itemize}
                \item DO/MD  (1)
                \item RN/PA  (2)
                \item Academic  (3)
                \item Analyst  (4)
                \item Student (Masters e.g., MPH)  (5)
                \item Student (MD/DO)  (6)
                \item Student (Nurse, PA)  (7)
                \item Student (Graduate)  (8)
                \item Student (Undergraduate)  (9)
                \item iTHRIV Scholar  (11)
                \item Other, please describe  (10)
            \end{itemize}

        \paragraph{Q2.6}

            What operating system will be on the computer you are using at the workshop or to participate in the online materials?

            \begin{itemize}
                \item Windows  (1)
                \item macOS  (2)
                \item Linux  (3)
                \item Not sure  (4)
            \end{itemize}

    \subsubsection{Programming Experience}

        \paragraph{Q3.1}

            In general, which of these best describes your experience with programming?

            \begin{itemize}
                \item I have none  (1)
                \item I took some programming related class in the past but have not used it since  (5)
                \item I have written a few lines now and again  (2)
                \item I have written programs for my own use that are a couple of pages long  (3)
                \item I have written and maintained larger pieces of software  (4)
            \end{itemize}

        \paragraph{Q3.2}

            What programming languages have you used in the past? Select all that apply.

            \begin{itemize}
                \item VBA (Visual Basic for Applications)  (1)
                \item Python  (2)
                \item R  (3)
                \item Perl  (4)
                \item Matlab  (5)
                \item Javascript  (6)
                \item C  (7)
                \item C++  (8)
                \item Fortran  (9)
                \item Other, please list  (10)
            \end{itemize}

        \paragraph{Q3.3}

            How familiar are you with interactive programming languages like Python or R?

            \begin{itemize}
                \item I do not know what those are  (1)
                \item I have heard of them but have never used them before  (2)
                \item I have installed it, but have only done simple examples with them  (3)
                \item I have written a small program with them before  (4)
                \item I use it to automate certain repetitive tasks  (5)
                \item I have small side projects that I program in it  (6)
                \item I program in them for work  (7)
            \end{itemize}

        \paragraph{Q3.4}

            How often do you currently use programming languages (R, Python, etc.)?

            \begin{itemize}
                \item Never  (1)
                \item Less than once per year  (2)
                \item Several times per year  (3)
                \item Monthly  (4)
                \item Weekly  (5)
                \item Daily  (6)
            \end{itemize}

        \paragraph{Q3.5}

            Which of these best describes how easily you could write a program (in any language) to find the largest number in a list?

            \begin{itemize}
                \item I wouldn’t know where to start  (1)
                \item I could struggle through, but not confident I could do it  (4)
                \item I could struggle through by trial and error with a lot of web searches  (2)
                \item I could do it quickly with little or no use of external help  (3)
            \end{itemize}

        \paragraph{Q3.6}

            How often do you currently use a specialized software with a point-and-click graphical user interface on your own
            (e.g., for statistical analysis: SPSS, SAS, …; for Geospatial analysis: ArcGIS, QGIS, … ; for Genomics analysis: Geneious, …)?

            \begin{itemize}
                \item Never  (1)
                \item Less than once per year  (2)
                \item Several times per year  (3)
                \item Monthly  (4)
                \item Weekly  (5)
                \item Daily  (6)
            \end{itemize}

        \paragraph{3.7}

            How often do you currently use Databases (SQL, Access, etc.)

            \begin{itemize}
                \item Never  (1)
                \item Less than once per year  (2)
                \item Several times per year  (3)
                \item Monthly  (4)
                \item Weekly  (5)
                \item Daily  (6)
            \end{itemize}

    \subsubsection{Data Cleaning and Processing Experience}

        \paragraph{4.1}

            How familiar are you with Microsoft Excel?

            \begin{itemize}
                \item I have never used it, or I have tried it but can't really do anything with it.  (1)
                \item I have used it as an electronic todo list and planner putting schedules and task deadlines in a single place  (2)
                \item I've used it to store datasets and able to calculate basic aggregate values, such as mean and sums  (3)
                \item I've used data aggregation, pivot tables, formulas, and plotting feature to understand how my data breaks down.  (4)
                \item I’ve coded up VBA macros and made VLOOKUP calls integrating multiple sheets for a simulation task  (5)
            \end{itemize}

        \paragraph{4.2}

            If you were given a dataset (e.g., Excel file, CSV file)
            and asked to do  some preliminary analysis on it,
            which of these best describe how easily you can accomplish the task?

            \begin{itemize}
                \item I wouldn't know where to start  (1)
                \item I could struggle through, but not confident I could do it  (4)
                \item I could struggle through by trial and error with a lot of web searches  (2)
                \item I could do it quickly with little or no use of external help  (3)
            \end{itemize}

        \paragraph{Q4.3}

            Are you familiar with the term ``tidy data''?

            \begin{itemize}
                \item I have never heard of the term  (1)
                \item I have heard of it but don’t remember what it is.  (2)
                \item I have some idea of what it is, but am not too clear  (3)
                \item I know what it is and could explain what it pertains to  (4)
            \end{itemize}

        \paragraph{Q4.4}

            Do you know what ``long'' and ``wide'' data are?

            \begin{itemize}
                \item I have never heard of the term  (1)
                \item I have heard of it but don’t remember what it is.  (2)
                \item I have some idea of what it is, but am not too clear  (3)
                \item I know what it is and could explain what it pertains to  (4)
            \end{itemize}

    \subsubsection{Project and Data Management}

        \paragraph{Q5.1}

            Please rate your level of satisfaction with your current data management and analysis workflow
            (e.g. how you collect, organize, store and analyze your data).

            \begin{itemize}
                \item Very unsatisfied  (1)
                \item Unsatisfied  (2)
                \item Neutral  (3)
                \item Satisfied  (4)
                \item Very satisfied  (5)
                \item Not sure  (6)
                \item Not applicable  (7)
                \item Never thought about this  (8)
            \end{itemize}

        \paragraph{Q5.2}

            Which of the following best describes how do you manage your data and analysis?

            \begin{itemize}
                \item I don't do data and/or analysis work  (1)
                \item My data and analysis are all in excel files, possibly with multiple sheets.  (2)
                \item I work on carefully time-stamped excel files for my version control and analysis  (3)
                \item I use some programming language to load in my data sets for analysis, but sometimes modify my original data files when cleaning the data  (4)
                \item I hold my original data sacred, and only work on it from another program and save out intermediate and final data projects as separate files  (5)
                \item I have a very specific project structure where data and analysis are kept in separate areas and have a version control system (e.g., Git, SVN)  (6)
                \item I have version controlled project templates along with build scripts (e.g., Makefile) to reproduce various aspects of the analysis  (7)
            \end{itemize}

    \subsubsection{Statistics}

        \paragraph{Q6.1}

            If you were given a dateset containing 2 cholesterol treatment options (drug and placebo),
            patients' baseline cholesterol values, and
            cholesterol values 4 weeks after treatment has started,
            would you know how to conduct a statistical analysis to see if there is a difference between the 2 groups?
            Any type of model will suffice.

            \begin{itemize}
                \item I wouldn't know where to start  (1)
                \item I could struggle through, but not confident I could do it  (4)
                \item I could struggle through by trial and error with a lot of web searches  (2)
                \item I could do it quickly with little or no use of external help  (3)
            \end{itemize}

        \paragraph{Q6.2}

            If you were given a dateset containing an individual's smoking status
            (binary variable) and whether or not they have hypertension (binary variable),
            would you know how to conduct a statistical analysis to see
            if smoking has an increased relative risk or odds of hypertension? Any
            type of model will suffice.

            \begin{itemize}
                \item I wouldn't know where to start  (1)
                \item I could struggle through, but not confident I could do it  (4)
                \item I could struggle through by trial and error with a lot of web searches  (2)
                \item I could do it quickly with little or no use of external help  (3)
            \end{itemize}

        \paragraph{Q6.3}

            If you were given a dateset comparing different treatment methods for
            cancer patients. Would you know how to conduct an analysis to see which
            treatment had a higher survival rate of patients?

                \begin{itemize}
                    \item I wouldn't know where to start  (1)
                    \item I could struggle through, but not confident I could do it  (4)
                    \item I could struggle through by trial and error with a lot of web searches  (2)
                    \item I could do it quickly with little or no use of external help  (3)
                \end{itemize}

        \paragraph{Q6.4}

            Are you familiar with the term "dummy variable"? It is sometimes also
            called "one-hot encoding".

            \begin{itemize}
                \item I have never heard of the term  (1)
                \item I have heard of it but don’t remember what it is  (2)
                \item I have some idea of what it is, but am not too clear  (3)
                \item I know what it is and could explain what it pertains to  (4)
            \end{itemize}

    \subsubsection{Workshop Framing and Motivation}

        \paragraph{Q7.1}

            Why are you participating in this workshop? Please check all that apply.

            \begin{itemize}
                \item To learn new skills  (1)
                \item To refresh or review my skills  (2)
                \item To learn skills that I can apply to my current work  (3)
                \item To learn skills that I can apply to my work in the future  (4)
                \item To learn skills that will help me get a job or a promotion  (5)
                \item As a requirement for my program or current position  (6)
            \end{itemize}

        \paragraph{7.2}

            Please rate your level of agreement with the following statements:

            \begin{itemize}
                \item Strongly Disagree (1)
                \item Disagree (2)
                \item Somewhat Disagree (3)
                \item Neither Agree nor Disagree (4)
                \item Somewhat Agree (5)
                \item Agree (6)
                \item Strongly Agree (7)
            \end{itemize}

            \begin{itemize}
                \item I believe having access to the original, raw data is important to be able to repeat an analysis. (1)
                \item I can write a small program, script, or macro to address a problem in my own work. (2)
                \item I know how to search for answers to my technical questions online. (3)
                \item While working on a programming project, if I got stuck, I can find ways of overcoming the problem. (4)
                \item I am confident in my ability to make use of programming software to work with data. (5)
                \item Using a programming language (like R or Python) can make my analyses easier to reproduce. (6)
                \item Using a programming language (like R or Python) can make me more efficient at working with data. (7)
            \end{itemize}

        \paragraph{7.3}

            Please share what you most hope to learn from participating in this workshop and/or workshop series.

        \paragraph{7.4}

            What do you want to know or be able to do after this workshop (or series of sessions) that you don't know or can't do right now?

\end{document}
