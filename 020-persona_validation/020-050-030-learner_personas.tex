\documentclass[020-persona\_validation.tex]{subfiles}

\begin{document}

\subsection{Learner Personas}
\label{sse:learner-personas}

    Three (3) learner personas were identified int he analysis.
    The initial wave of results had four (4) learner personas,
    but as more data was collected, Patricia Programmer
    (already competent programmers)
    was was dropped as the results were combined with Samir Student.

    \subsection{Ash Academic}

        Alex, a professor of bioinformatics,
        studies molecular dynamics of proteins and protein-protein interactions.
        They are also responsible for teaching an introductory research class to 100 freshman and sophomore students every year.
        They also run a data consulting service at their university,
        providing support for data-related challenges to research and instructors from any discipline.

        Their students complain that intro courses in programming are too theoretical and
        require more programming knowledge than they have.
        Many students in the department also cannot register for similar classes.
        Alex, has 10s of students working for them in computational molecular dynamics simulations and other data analytics projects.

        \subsubsection{Relevant prior knowledge or experience}

            Alex performs their research using a combination of
            Excel spreadsheets and specialized software.
            But the are starting to move to using R or Python (which they taught themselves during a sabbatical).
            They have never taken a formal programming course,
            and suffers from impostor syndrome in discussions about programming.
            Alex would like to learn more about how programming can help their research and
            keep up with the tools their students are learning in class.
            They also need resources to give to their student researches to help work on projects in the lab.

        \subsubsection{Perception of needs}

            Alex needs workshops and how-to guides so they can better block off time and apply new skills to their research.
            They would like ready-to-use lesson material that they can learn from and also be be remixed for their students.
            The materials should be at in introductory level demystify jargon (e.g., what is ``tidy data''?).
            Alex also wants to be able to teach the same materials they use for their own research
            to amortize learning costs and stay in practice.

        \subsubsection{Special considerations}

            Alex wants to provide technical training to students,
            but does not have the actual time to teach all the relevant skills.
            As a person in STEM,
            they typically find themselves isolated and alone when taking formal technical classes and
            is scared to appear ignorant, and are reluctant to speak up and ask questions.


    \subsection{Clare Clinician}

        Clare has spent the last 6 years working in the Cardiothroasic ICU in a large medical hospital system.
        They see the impacts of data science in their day-to-day job
        and want to learn these skills to better understand their patients.
        However, nothing makes sense when trying to learn it on their own,
        and there are few formal opportunities to learn that fit into their work schedule.
        Clare has always been a good student and always excelled at things they tried to learn;
        they are hard on themselves when struggling to learn a new skill and
        would rather place blame on the long hours at work than having their peers know they could use assistance.

        \subsection{Relevant prior knowledge or experience}

            Clare keeps up with medical research,
            but has little to no experience in doing medical research.
            They use Excel for non-data related tasks (e.g., making lists),
            or manually inputting patient data into spreadsheets for chart reviews.
            Clare wants to be able to collect and manage data as well as
            learn about the process behind data analysis to perform their own analysis and study one day.

        \subsection{Perception of needs}

            Clare wants self-paced tutorials with practice exercises that uses health related data they can directly relate to.
            Since programming and data analysis are relatively new to them,
            Since they are new to programming,
            a class that provides overviews to orient them to programming and ask questions along with
            a community forum where they can ask for help asynchronously is needed.
            The introductory tutorials should contain detailed visualizations on how to install, setup, and use
            the tools they will learn on datasets that they can relate to.
            They do not mind having references to other materials,
            but do want a single learning path.

        \subsection{Special considerations}

            Clare is a single parent who juggle their time at work and at home who are strapped for time to learn a new skill.
            They are typically only able to take additional classes early in the morning or late at night,
            outside of ``normal'' work hours.

    \subsection{Samir Student}

        Samir is a graduate student in a bioinformatics program.
        They worked in a wet lab during their undergraduate days studying neuroscience.
        These days, Samir is doing more computational work and
        starting to use programming based tools to look at protein structures with Ash Academic.
        They've taken a few classes that had had programming based homework assignments and projects,
        but the lectures themselves were mostly around theory, and many of the programming skills were self-taught.

        \subsubsection{Relevant prior knowledge or experience}

            Samir is fairly proficient in Excel,
            does works with spreadsheets regularly, and
            knows how to load up Excel spreadsheets into R and do basic data processing and analysis.
            However,
            they do not have much programming practice outside of classroom homeworks and projects.
            They spend a lot of their time on StackOverflow copying and pasting code
            so they don't consider themselves a ``real programmer''.
            They have no problem getting their work done,
            but usually involves a lot of googling to eventually get the solution.

        \subsubsection{Perception of Needs}

            Samir wants a formal workshop and reference materials
            that can be used to build a good foundation of the programming skills they were never taught.
            They want a better understanding of the terminology and jargon used in data science
            so they have the vocabulary to search for and understand solutions posted online.
            They are also looking for a community to help in their growth as a student in this domain,
            and also materials they can pass on to the more junior researchers in the lab.

        \subsubsection{Special considerations}

            Samir has a disability (vision, hearing, attention, etc)
            that make it difficult to learn in ``traditional'' classroom settings.
            They typically need a recording of the lectures so they can rewatch lectures at their own pace.

\end{document}
