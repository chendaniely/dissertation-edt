\documentclass[020-persona\_validation.tex]{subfiles}

\begin{document}

    % In this chapter/paper,
    % I confirm personas can be empirically created used in education
    % and demonstrate that they can also be empirically created for learners (students).
    % I demonstrate this process by creating and validating a survey where I then cluster responses together
    % The purpose of the work is to demonstrate that you are able to
    % create learner personas in the a specific domain (e.g., biomedical sciences) in order to
    % create more relevant and domain-specific data science learning materials.
    % this is written in a formal and analytic tone for educators wishing to create lesson materials

    % intro outline:
    % why make data science materials specific? people learn better.
    % are there reasons other domains want to use it? data science cuts across all fields
    % how would you go about making domain-specific materials? make learner personas
    % what are learner personas? they are fictional characters that represent a portion of your audience
    % how would you create them? i use a survey to cluster them
    % how do you know the survey is good? i also validate the surveys.
    % How does this help create lesson materials?

    The abundance of data science learning materials have made it a commodity
    \cite{krossDemocratizationDataScience2020}.
    The massive online open courses (MOOCs) providers like
    Coursera,
    Udacity, and
    EdX
    offer dozens of free data sciences courses
    (57, 7, and 68, respectively)
    \cite{courseraTopFreeCourses, udacityDataScienceOnline, edxDataAnalysisCourses}.
    Publishing tools like Bookdown and
    JupyterBook
    also catalogue free online book resources
    (1075 and 63, respectively)
    \cite{bookdownAllBooksBookdown, executablebookprojectGalleryJupyterBooks}.
    However, many of these resources are geared towards general populations or general topics,
    and examples are not always relevant to new learners
    \cite{krossDemocratizationDataScience2020}.
    For working practitioners we also need to consider their needs and time constraints.

    There are community projects that try to create compilations of more resources,
    e.g., ``The Big Book of R'' has 267 free books on R programming with
    21 books under its life sciences section, and
    only 4 resources with a medical or epidemiology focus,
    one of them is the work the authors are developing \cite{baruffaBigBook2021}).
    Instead of creating data science materials for general audiences,
    there is the opportunity to create more domain-specific lesson materials
    with a direct focus on pedagogy,
    identifying more data science concepts,
    and relevant examples
    \cite{krossDemocratizationDataScience2020}.
    In the medicine, the demand for informatics courses outpaces and exceeds the training opportunities
    \cite{banerjeeMedicalStudentAwareness2015, americanmedicalassociationStudentInterestInformatics, americanmedicalassociationAcceleratingChangeMedical2021, americanmedicalassociationEducation}.
    Increasing the quantity, quality, and publicity of medically focused
    data science materials can alleviate the training and opportunity gap
    \cite{banerjeeMedicalStudentAwareness2015, americanmedicalassociationAcceleratingChangeMedical2021}.
    More relevant and more content-accessible lesson materials often leads to better engagement and motivation from learners
    \cite{wilson2019teaching, ambrose2010learning, Koch2016}.
    

    % Dedicated courses on data products, data cleaning, reproducible science, and exploratory analysis
    % are typically lacking in data science programs.
    % But these processes represent most of the data science workflow.
    % It can be argued that these topics can be folded into existing
    % statistical inference, modeling, and programming courses,
    % but those are not always topics and courses that everyone who works with data
    % wants and/or needs to learn.
    % Preparing the data, being able to explore it is sometimes enough to gain better sites,
    % especially when paired with domain expertise.

    Online courses, including massive open online courses (MOOCs),
    do have the flexibility to create
    domain-specific and focused learning materials.
    But the lack of clinical informatics training and mentoring opportunities available to medical students
    suggest that there is still a need to increase the quantity, quality, and publicity of
    data science materials catered towards the biomedical sciences
    \cite{banerjeeMedicalStudentAwareness2015}.
    In higher education,
    new data science programs are usually created with joint departments,
    typically computer science and statistics
    \cite{songBigDataData2016}.
    Depending on how course registration and prerequisites are implemented and specified,
    joint departments can alienate and create a barrier of entry to students who are not already
    in one of the the joint departments
    \cite{kelleherLoweringBarriersProgramming2005}.
    It is not surprising that many bachelor's programs place a heavy emphasis on
    courses in the computer science and statistics departments
    \cite{songBigDataData2016}.
    Domain-specific data science learning materials can not only lower the barrier of entry of learning,
    but the more common data science workflow steps of
    data cleaning, reproducible science, exploratory analysis, and creating data products,
    can be optimized for the needs of domain experts and make
    data science skills more accessible to all departments and domains,
    not just withing statistics and computer science
    \cite{krossDemocratizationDataScience2020, cc2020, ccdsc2021, payneBiomedicalInformaticsMeets2018}.

    % Data literacy relates to all the competencies and steps needed to work with data,
    % and communicate findings.
    % Typically this involves more than finding a single clean dataset and running aggregate statistics,
    % and requires more exploratory data analysis (EDA) skills.
    % For working professionals in a specific domain (e.g., medicine, law, business, etc)
    % many of the highly technical statistics and computer science courses may be too specific for their needs,
    % and also will take too long to build up the necessary prerequisite knowledge to get to that point.
    % At the same time, these working professionals have immediate data needs,
    % whether that be curating a dataset manually or tweaking an already existing analysis.

    Healthcare providers (e.g., medical doctors, veterinarians, nurses, physician assistants, other clinicians, and administrators)
    play a key role in the interdisciplinary needs in health care
    \cite{surowiecki2005wisdom, hoytOverviewTwoOpen2018, payneBiomedicalInformaticsMeets2018}.
    Being able to work with open collaborative data science platforms,
    the healthcare domain-experts are able to better understand, utilize, and contribute to the
    ``wisdom of crowds''
    \cite{surowiecki2005wisdom, hoytOverviewTwoOpen2018, payneBiomedicalInformaticsMeets2018}.
    A coordinated effort between the biomedical communities and data science communities
    is a priority in order to create effective curricular frameworks for mutual understanding
    of each respective domain
    \cite{payneBiomedicalInformaticsMeets2018}.

    In the United states,
    National Institute of Health (NIH) is responsible for biomedical and public health research.
    They have made a strategic data science plan to improve the
    storage, management, standardization and publication of biomedical research.
    Professional organizations such as the
    American Medical Association (AMA),
    American Nurses Association (ANA),
    American Academy of Physician Associates (AAPA), and
    American Veterinary Medical Association (AVMA)
    serve as national organizations for medical doctors, nurses, physician associates, and veterinarians.
    All of these organizations have made calls for the importance of data science in their respective professions
    \cite{payneBiomedicalInformaticsMeets2018, americanmedicalassociationAcceleratingChangeMedical2021, americannursesassociationANAEnterpriseAmerican, owenEthicalIntersectionHealthcare2017, nolenArtificialIntelligenceVeterinary2020, nationalinstitutesofhealthNIHStrategicPlan2020}.
    The NIH's strategic data science plan has five (5) main goals and objectives:
    (1) data infrastructure,
    (2) modernized data ecosystem,
    (3) data management, analysis, and tools,
    (4) workforce development, and
    (5) stewardship and sustainability. % TODO insert image
    The objectives in
    data management, analytics, and tools;
    workforce development;
    and stewardship and sustainability
    relate to the more individual data science skills that integrate with larger data infrastructure
    \cite{nationalinstitutesofhealthNIHStrategicPlan2020}.

    % The descriptions of all these goals do not make an explicit call for statistical methods or computer algorithms,
    % rather, they are mainly involved in the management of data and how it integrates into existing ecosystems
    % for Findability, Accessibility, Interoperability, and Reuse (FAIR) of digital assets.
    % Having a better understanding of data literacy concepts,
    % will help develop a research workforce.

    % The American Medical Association (AMA) has underlined the importance of data skills and data science
    % as the new frontier for physician training.
    % The American Nursing Association (ANA) has also talked about using data to improve patient care.
    % In fact, interest in informatics among medical students outpace opportunities.
    % There is a lack of approachable data science materials,
    % and increasing the quantity, quality, and publicity all work towards meeting these opportunity shortages.
    

    This work seeks to utilize best practices in pedagogical design and create domain-specific,
    learner-informed biomedical data science education materials.
    Persona methodologies help identify our learner's prior knowledge and perception of needs
    that are grounded in empirical data
    to create relevant domain-specific learning materials
    \cite{pruittPersonaLifecycleKeeping2006, zagallo2019through, schwartzParadoxChoiceWhy2016}.

    Persona methodology is a systematic way to collect data on our potential learners,
    and identify key components of our target audience.
    These ``learner personas'' are fictional characters based on empirical data that represent key characteristics
    of our learners.
    This is a technique is commonly used by the design industry to bridge the gap between product designers and product users
    \cite{pruittPersonaLifecycleKeeping2006, zagallo2019through, schwartzParadoxChoiceWhy2016}.
    Here, we apply these methods to the educational space,
    identifying what our learners know and how to build on their existing knowledge.

    Persona methodologies have been used to identify
    college instructors and teacher professional development needs
    \cite{zagallo2019through}.
    While learner personas have been created in the past to help focus educational materials
    \cite{RStudio2019, softwarecarpentryLearnerProfiles},
    the process, data, and survey instruments, were not published.
    This study aims to use a data and analysis driven approach on creating learner personas for the biomedical sciences
    in developing relevant data science materials to people who
    work in the biomedical space.
    These personas provide a more memorable character and also serve to reduce the cognitive load of instructors
    when preparing and tailoring lesson content since all the considerations are incorporated into specific characters,
    and not disparate lists
    \cite{pruittPersonaLifecycleKeeping2006, schwartzParadoxChoiceWhy2016, cooperInmatesAreRunning1999}.
    There are for (4) main benefits of using the persona methodology, they:
    (1) make assumptions about users explicit,
    (2) place the focus on specific types of users rather than on all possible users,
    (3) help us make better decisions by limiting choices, and
    (4)Engage the product design and development team
    \cite{pruittPersonaLifecycleKeeping2006, schwartzParadoxChoiceWhy2016}.
    We create personas to understand what our learners know and how to get them competent and confident
    in performing data literacy tasks and basic data analysis techniques.

    % Little, if any, have been published about students and learners,
    % and they are nonexistent for data science students in the biomedical sciences.
    % Existing personas in data science are not accompanied with a published study.
    % This study builds on learner personas created in the data science field that
    % incorporate learners' backgrounds, relevant prior knowledge and experience, perception of needs, and
    % special considerations.

    One of the secondary goals of this study is to validate the survey used in persona identification,
    and have it be a useful tool for other educators to identify how much
    programming, data literacy, and technical background their potential learners will have,
    to create a better learning experience for the learners.
    We can use the learner persona information with concept maps along with a backwards design
    to create new lesson content.

    % For any given lesson topic,
    % creating concept maps or doing a task deconstruction exercise,
    % help identify all the component knowledge and how they are connected with one another
    % a useful exercise for instructors is creating a
    % concept maps or  as methods to identify how individual topics are related to one another.

    % For learners, identifying how many connections they are familiar with can aid in identifying
    % which topics need to be covered.
    % The density of connections also point words their abilities from novice, competent, proficient, expert, and master on the
    % Dreyful model of skill acqusition.

\end{document}
