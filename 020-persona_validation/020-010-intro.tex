\documentclass[020-persona\_validation.tex]{subfiles}

\begin{document}

    % In this chapter/paper,
    % I confirm personas can be empirically created used in education
    % and demonstrate that they can also be empirically created for learners (students).
    % I demonstrate this process by creating and validating a survey where I then cluster responses together
    % The purpose of the work is to demonstrate that you are able to
    % create learner personas in the a specific domain (e.g., biomedical sciences) in order to
    % create more relevant and domain-specific data science learning materials.
    % this is written in a formal and analytic tone for educators wishing to create lesson materials

    % intro outline:
    % why make data science materials specific? people learn better.
    % are there reasons other domains want to use it? data science cuts across all fields
    % how would you go about making domain-specific materials? make learner personas
    % what are learner personas? they are fictional characters that represent a portion of your audience
    % how would you create them? i use a survey to cluster them
    % how do you know the survey is good? i also validate the surveys.
    % How does this help create lesson materials?

    The abundance of data science learning materials have made it a commodity.
    However, many of these resources are geared towards general populations,
    and examples are not always relevant to new learners.
    More relevant lesson materials lead to better engagement and motivation from learners.
    Instead of creating data science materials for general audiences,
    there is the opportunity to create more domain-specific lesson materials.

    % Dedicated courses on data products, data cleaning, reproducible science, and exploratory analysis
    % are typically lacking in data science programs.
    % But these processes represent most of the data science workflow.
    % It can be argued that these topics can be folded into existing
    % statistical inference, modeling, and programming courses,
    % but those are not always topics and courses that everyone who works with data
    % wants and/or needs to learn.
    % Preparing the data, being able to explore it is sometimes enough to gain better sites,
    % especially when paired with domain expertise.

    New data science programs are usually created with joint departments,
    typically computer science and statistics.
    Depending on how course registration and prerequisites are implemented and specified,
    joint departments can alienate and create a barrier of entry to students who are not already
    in one of the the join departments.
    It is not surprising that many bachelor's programs place a heavy emphasis on
    courses in the computer science and statistics departments.
    But, computational needs exist across many, if not all, departments.

    % Data literacy relates to all the competencies and steps needed to work with data,
    % and communicate findings.
    % Typically this involves more than finding a single clean dataset and running aggregate statistics,
    % and requires more exploratory data analysis (EDA) skills.
    % For working professionals in a specific domain (e.g., medicine, law, business, etc)
    % many of the highly technical statistics and computer science courses may be too specific for their needs,
    % and also will take too long to build up the necessary prerequisite knowledge to get to that point.
    % At the same time, these working professionals have immediate data needs,
    % whether that be curating a dataset manually or tweaking an already existing analysis.

    As the primary agency in the United States responsible for biomedical and public health research,
    the National Institute of Health (NIH) has created a strategic plan for data science with five (5)
    main goals and objectives:
    (1) data infrastructure,
    (2) modernized data ecosystem,
    (3) data management, analysis, and tools,
    (4) workforce development, and
    (5) stewardship and sustainability. % TODO insert image
    The objectives in
    data management, analytics, and tools;
    workforce development;
    and stewardship and sustainability
    relate to the more individual data science skills that integrate with larger data infrastructure.

    % The descriptions of all these goals do not make an explicit call for statistical methods or computer algorithms,
    % rather, they are mainly involved in the management of data and how it integrates into existing ecosystems
    % for Findability, Accessibility, Interoperability, and Reuse (FAIR) of digital assets.
    % Having a better understanding of data literacy concepts,
    % will help develop a research workforce.

    The American Medical Association (AMA) has underlined the importance of data skills and data science
    as the new frontier for physician training.
    The American Nursing Association (ANA) has also talked about using data to improve patient care.
    In fact, interest in informatics among medical students outpace opportunities.
    There is a lack of approachable data science materials,
    and increasing the quantity, quality, and publicity all work towards meeting these opportunity shortages.

    The authors aimed to create a set of data science materials for clinicians and practitioners in medicine.
    We use persona methodologies to identify our learner's prior knowledge and perception of needs
    to create relevant domain-specific learning materials,

    Persona methodology is a systematic way to collect data on our potential learners,
    and identify key components of our target audience.
    These ``learner personas'' are fictional characters based on empirical data that represent key characteristics
    of our learners.
    This is a technique used by the design industry to bridge the gap between product designers and product users.
    Here, we apply these methods to the educational space,
    identifying what our learners know and how to build on their existing knowledge.

    Persona methodologies have been used to identify
    college instructors and teacher professional development needs.
    However, we can use the techniques in previous persona studies
    for our current work in developing relevant data science materials to people who
    work in the biomedical space.
    These personas provide a more memorable character and also serve to reduce the cognitive load of instructors
    when preparing and tailoring lesson content since all the considerations are incorporated into specific characters,
    and not disparate lists.
    We create personas to understand what our learners know and how to get them competent and confident
    in performing data literacy tasks and basic data analysis techniques.

    % Little, if any, have been published about students and learners,
    % and they are nonexistent for data science students in the biomedical sciences.
    % Existing personas in data science are not accompanied with a published study.
    % This study builds on learner personas created in the data science field that
    % incorporate learners' backgrounds, relevant prior knowledge and experience, perception of needs, and
    % special considerations.

    One of the secondary goals of this study is to validate the survey used in persona identification,
    and have it be a useful tool for other educators to identify how much
    programming, data literacy, and technical background their potential learners will have,
    to create a better learning experience for the learners.
    We can use the learner persona information with concept maps along with a backwards design
    to create new lesson content.

    % For any given lesson topic,
    % creating concept maps or doing a task deconstruction exercise,
    % help identify all the component knowledge and how they are connected with one another
    % a useful exercise for instructors is creating a
    % concept maps or  as methods to identify how individual topics are related to one another.

    % For learners, identifying how many connections they are familiar with can aid in identifying
    % which topics need to be covered.
    % The density of connections also point words their abilities from novice, competent, proficient, expert, and master on the
    % Dreyful model of skill acqusition.

    There are a few existing parameters we know going into this study.
    Our target audiance are
    (1) adult working professionals,
    (2) work as a clinician and/or do research in the biomedical field.
    (3) Excel is the main, if only, program for data work.
    These parameters already put in constraints into the lesson development phase:
    (1) The lesson needs to be modular and sections need to be taught within 1 hour blocks.
    This can account for busy work schedules and can possibly be given during break times (e.g., lunch).
    (2) Written lesson materials need to be provided online for asynchronous learning and can be used as reference material later on.
    (3) Classes will be recorded and posted online for the same reason written materials are posted.
    (4) Data examples must be medically or health realated.
    (5) A separate lesson needs to be given about working with data in Excel and setting up spreadsheet data.
    (6) An example of loading an Excel spreadsheet into the programming language so learners can import any existing datasets.
    A separate study will build on these concepts to create concrete learning objectives (LOs) for the lesson materials.
    These studies are learner focuced and does not focus on the skills needed to properly present the materials as an instructor.
    The Carpentries Instructor Training materials is a good reference to learn more about evidence-based teaching practices.
    But, this study will provide a foundation for future instructional material creators identify core data literacy components
    that need to be taught.

\end{document}
