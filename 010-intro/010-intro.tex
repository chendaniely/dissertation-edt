\documentclass[../main.tex]{subfiles}
\begin{document}

    \subfile{010-010-intro}

    \section{History of Data Science}
        \label{se:intro-ds-history}

        Statistics, data science, and computational programming education
        have co-evolved in the last half-century.

        \subfile{010-020-ds_history}
        \subfile{010-030-r_history}
        \subfile{010-040-python_history}
        \subfile{010-045-emerging_technologies}
        \subfile{010-050-swc_history}

    \section{Data Science Education and Pedagogy Research}
        \label{se:intro-ds-edu-ped}

        Computing and statistics education have both created their own sets of
        higher education curriculum guidelines.
        Many of these concepts have made their way down from higher education to K-12,
        suggesting the top-down need for learning data science skills.

        \subfile{010-060-computing_education}
        \subfile{010-080-statistics_education}
        \subfile{010-070-data_science_education}

    \section{Learner Personas and Pedagogical Backed Strategies for Creating Accessible Content in Data Science}
        \label{se:intro-personas}

        \subfile{010-110-personas}

    \section{Best Practices in Teaching Data Science}
        \label{se:intro-teaching-best-practices-ds}

        \subfile{010-120-best_practices}

    \section{The Need for Pedagogically-Backed Data Science Curriculum in Medicine}
        \label{se:intro-ds-edu-gaps}

        \subfile{010-130-domain-ds-materials}

    \section{Building a Community (of Practice)}

        This dissertation does not explicitly explore the process of online community building,
        but this is an important idea to keep in mind when creating educational materials,
        as connecting with other educators to form a community of practice
        where (typically geographically co-located) people with a common set of goals, interests, and concerns
        can learn from one another
        \cite{wilson2019teaching}.
        Organizations like The Carpentries can provide online teaching communities of practice
        where other instructors can learn from one another around
        builing the pedagogical content knowledge for teaching, not just around data science
        \cite{CarpentriesHowWe, shulmanThoseWhoUnderstand1986}.

        Community building is a slow process, and there are four (4) main components on building and sustaining a community.
        The first step is onboarding and recruiting new members.
        A Code of Conduct should be prominent during this phase of community building and ensure members are in a safe environment.
        Onboarding should also include resources to bring new members up to date on current community events.
        The second step is around retention.
        Community members should have some sense of agency and be able to contribute to the group,
        and these contributions should be acknowledged.
        As communities grow, eventually a governance model will need to be created to help
        provide top-down guidance for major decisions.
        These governance models do not need to be strictly hierarchical,
        and can adapt with the size of the community.
        Finally, retention and onboarding is a never-ending process.
        Healthy communities plan for the efflux of members by onboarding more members and helping to retain them.

        This dissertation primarily focuses on the learners, and finding ways to improve their learning.
        Teaching is an art and skill on its own
        \cite{greenBuildingBetterTeacher2014, wilson2019teaching}.
        Knowing what needs to be taught (content knowledge),
        how the materials are taught (pedagogical content knowledge), and
        why topics are taught in context (curricular knowledge)
        are all aspects of teaching knowledge that instructors can benefit from joining a teaching
        community of practice
        \cite{shulmanThoseWhoUnderstand1986, greenBuildingBetterTeacher2014, wilson2019teaching}.

    % \section{Ethics}


    % \cite{Chen2020}


    % \cite{ostblomOpinionatedPracticesTeaching2021}

    % https://journalofethics.ama-assn.org/article/ethical-dimensions-using-artificial-intelligence-health-care/2019-02

    % https://www.sciencedirect.com/science/article/pii/S0933365715000871?via%3Dihub

    % https://www.ncbi.nlm.nih.gov/pmc/articles/PMC7332220/

    % https://med.stanford.edu/news/all-news/2018/03/researchers-say-use-of-ai-in-medicine-raises-ethical-questions.html

    % https://www.theverge.com/2021/6/30/22557119/who-ethics-ai-healthcare


\end{document}
