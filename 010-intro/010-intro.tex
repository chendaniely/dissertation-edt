\documentclass[../main.tex]{subfiles}
\begin{document}

    \subfile{010-010-intro}

    \section{History of Data Science}
        \label{se:intro-ds-history}

        Statistics, data science, and computational programming education
        have co-evolved in the last half-century.

        \subfile{010-020-ds_history}
        \subfile{010-030-r_history}
        \subfile{010-040-python_history}
        \subfile{010-045-emerging_technologies}
        \subfile{010-050-swc_history}

    \section{Data Science Education and Pedagogy Research}
        \label{se:intro-ds-edu-ped}

        Computing and statistics education have both created their own sets of
        higher education curriculum guidelines.
        Many of these concepts have made their way down from higher education to K-12,
        suggesting the top-down need for learning data science skills.

        \subfile{010-060-computing_education}
        \subfile{010-080-statistics_education}
        \subfile{010-070-data_science_education}

    \section{Learner Personas and Pedagogical Backed Strategies for Creating Accessible Content in Data Science}
        \label{se:intro-personas}

        \subfile{010-110-personas}

    \section{Best Practices in Teaching Data Science}
        \label{se:intro-teaching-best-practices-ds}

        \subfile{010-120-best_practices}

    \section{The Need for Pedagogically-Backed Data Science Curriculum in Medicine}
        \label{se:intro-ds-edu-gaps}
        
        Industry and government can play a special role in generating modern programs through
        professional advisory boards, work-study programs, and internships.
        Academic institutions must also be proactive in
        supporting strong, contemporary computing programs for the benefit of its graduates. [@cc2020]

        ``Many students of data science
        go on to teach, but it is rare to find a course in university cumcula on pedagogy.
        A
        rigorous evaluation
        of tools and their development applies to data science that which statisticians routinely advocate for
        process improvement in other disciplines'' [@clevelandDataScienceAction2001]

        Domain specific materials is what's missing

        Incorporating computing skills and domain knowledge can be thought of
        $\text{computing} + x$ and $x + \text{computing}$ (where $x$ is a knowledge domain)
        \cite{cc2020}.
        In `computing + x`, computing systems extend to non-computing disciplines.
        These fields usually have "informatics" in the term
        e.g., medical infomatics, bioinformatics, health informatics, legal informatics, etc [@CC2020].
        In `x + computing`,
        computing systems are extensions to an already existing and established field of study.
        One promonent example is in computational biology,
        where established laboratory methods expanded to computing.
        Both mechanisms of combining computing and a discipline allow for the discovery of transformatinal relationships,
        only thst starting point is different [@CC2020].

        [@shortliffe1993adolescence]
        The creation of this kind of infrastructure will require vision and resources from leaders who realize that the practice of medicine is inherently an information-management task and that biomedicine must make the same kind of coordinated commitment to computing technologies as have other segments of our society in which the importance of information management is well understood.

        This dissertation focus around the core and fundamental skills
        of computing, and how computing skills can better areas in the biomedical sciences (`x + computing`).
        By combining domain, computing, and integrative knowledge and skills,
        non-computing individuals can make the connections from their domain to the transformative opporunities
        created by using computing [@CC2020].


    \section{Building Community}

        Starting, building, maintaining a community.

    \section{Ethics}

    https://journalofethics.ama-assn.org/article/ethical-dimensions-using-artificial-intelligence-health-care/2019-02

    https://www.sciencedirect.com/science/article/pii/S0933365715000871?via%3Dihub

    https://www.ncbi.nlm.nih.gov/pmc/articles/PMC7332220/

    https://med.stanford.edu/news/all-news/2018/03/researchers-say-use-of-ai-in-medicine-raises-ethical-questions.html

    https://www.theverge.com/2021/6/30/22557119/who-ethics-ai-healthcare


\end{document}
