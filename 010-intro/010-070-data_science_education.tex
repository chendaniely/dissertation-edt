\documentclass[010-intro.tex]{subfiles}

\begin{document}

\subsection{Data Science Education (Higher Education)}

    The permeation of data science into fields other than statistics and into other fields,
    show how valuable it is for understanding the world
    \cite{clevelandDataScienceAction2001}.
    The spike in demand for data science skills in industry led way to the creation of
    ``Bootcamps''.
    These filled the initial void of learning data science skills,
    while the educational system caught up to train new graduates for the work environment.
    followed by combining departments and programs in higher education
    \cite{krossDemocratizationDataScience2020},
    and impacting K-12 education standards
    \cite{csta, gaise2k12}.
    The Bureau of Labor Statistics (BLS)
    predicts that by 2030 computational employment will increase by 13\% in the United States,
    faster than other occupations
    \cite{cc2020, blsOccupationalOutlook}.

    Data science is a set of fundamental concepts that involves
    principles, processes, and techniques to gain knowledge from data, typically using programming as the tool.
    \cite{cc2020, ccdsc2021, provostDataScienceBusiness2013}.
    It is a sub-domain of computing education that is adjacent to data analytics and data engineering.
    There are 11 other topical sub-domains that are incorporated into the core data science knowledge areas (KAs)
    \cite{cc2020, ccdsc2021}:
    (1) Analysis and Presentation (AP),
    (2) Artificial Intelligence (AI),
    (3) Big Data Systems (BDS),
    (4) Computing and Computer Fundamentals (CCF),
    (5) Data Acquisition, Management, and Governance (DG),
    (6) Data Mining (DM),
    (7) Data Privacy, Security, Integrity, and Analysis for Security (DP),
    (8) Machine Learning (ML),
    (9) Professionalism (PR),
    (10) Programming, Data Structures, and Algorithms (PDA), and
    (11) Software Development and Maintenance (SDM).
    Additionally, a full curriculum should be augmented with six (6) fundamental courses in
    mathematics, statistics, and computer science
    \cite{cc2020, ccdsc2021}:
    (1) Calculus,
    (2) Discrete structures,
    (3) Probability theory,
    (4) Elementary statistics,
    (5) Advanced topics in statistics, and
    (6) linear algebra.

    These suggestions pertain to data science degree programs in higher education
    \cite{cc2020, ccdsc2021},
    however, this dissertation focuses on introductory materials for working professionals in the biomedical sciences
    (e.g., clinicians, analysts, academics).
    There is simply not enough time to cover all the topics as suggested in the
    Computing Curriculum 2020 report and the Data Science Curricula 2021 report.
    The goal is not to train working domain experts (e.g., clinicians) to do all the work of a data scientist,
    rather give them the requisite knowledge and skills to work in multidisciplinary data science teams
    where they can better utilize their domain expertise.
    More about domain-specific data science education in adults are described in Section \ref{se:intro-domain-specificity}.

    % TODO TODO: find some materials about how much is needed for domain experts on data science teams.
    % Not all of this information is needed to be learned]
    
\subsection{Data Science Education (K-12)}

\end{document}
