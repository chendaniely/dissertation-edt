\documentclass[010-intro.tex]{subfiles}

\begin{document}

Industry and government can play a special role in generating modern programs through
professional advisory boards, work-study programs, and internships.
Academic institutions must also be proactive in
supporting strong, contemporary computing programs for the benefit of its graduates. [@cc2020]

``Many students of data science
go on to teach, but it is rare to find a course in university cumcula on pedagogy.
A
rigorous evaluation
of tools and their development applies to data science that which statisticians routinely advocate for
process improvement in other disciplines'' [@clevelandDataScienceAction2001]

Domain specific materials is what's missing

Incorporating computing skills and domain knowledge can be thought of
$\text{computing} + x$ and $x + \text{computing}$ (where $x$ is a knowledge domain)
\cite{cc2020}.
In `computing + x`, computing systems extend to non-computing disciplines.
These fields usually have "informatics" in the term
e.g., medical infomatics, bioinformatics, health informatics, legal informatics, etc [@CC2020].
In `x + computing`,
computing systems are extensions to an already existing and established field of study.
One promonent example is in computational biology,
where established laboratory methods expanded to computing.
Both mechanisms of combining computing and a discipline allow for the discovery of transformatinal relationships,
only thst starting point is different [@CC2020].

[@shortliffe1993adolescence]
The creation of this kind of infrastructure will require vision and resources from leaders who realize that the practice of medicine is inherently an information-management task and that biomedicine must make the same kind of coordinated commitment to computing technologies as have other segments of our society in which the importance of information management is well understood.

This dissertation focus around the core and fundamental skills
of computing, and how computing skills can better areas in the biomedical sciences (`x + computing`).
By combining domain, computing, and integrative knowledge and skills,
non-computing individuals can make the connections from their domain to the transformative opporunities
created by using computing [@CC2020].

\end{document}
