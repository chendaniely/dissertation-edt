\documentclass[010-intro.tex]{subfiles}

\begin{document}

\subsection{Emerging Technologies}

Data science's reach goes beyond any single domain and technology stack.
The Julia language reas releasted in 2012 with the goal of being a high-level and fast programing language,
especially around numerical computing.
It was incoporated in the Jupyter ecosystem as a way to unify computational langauges into a single
development enviornment for data science tasks.
Julia's \code{dataframes.jl} library was released in 2012, giving the same data manipulating
features as the R dataframe object and Python's \code{pandas} library.

Representing a dataframe object across multiple programming languages is a challenge because of
how array objects are implement from language to language.
The Apache Arrow project aims to create a centralized API for dataframes that can be used across mutliple languages,
alowing end users to use whatever tool best suits their needs,
but using a unified and performant data structure
\cite{ApacheArrow}.
Groups like Ursa Labs were formed in 2018 (now Voltron Data in 2021) to help provide
more support to the Apache Arrow project
\cite{UrsaLabs, VoltronData}.

JavaScript's popularity over the years stem from it connection with end user interactions with
virtually all websites on the internet.
It has become a tool for creating interactive figures (e.g., \code{D3.js}) for people on the internet.
Since virtual every person who connects to the internet uses a web browser,
WebAssembly was announced in 2015 as an open standard for any programming language
to compile down for web applications
\cite{WebAssembly}.

Projects like Apache Arrow and WebAssembly are blurring the lines between tools needed for data science tasks.

\end{document}
