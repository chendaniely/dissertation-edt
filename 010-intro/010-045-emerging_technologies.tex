\documentclass[010-intro.tex]{subfiles}

\begin{document}

\subsection{Emerging Technologies}

    The reach of data science goes beyond any single domain and technology stack.
    The Julia language (Rank \#36 on the TIOBE index in November 2021)
    was released in 2012 with the goal of being a high-level and fast programming language,
    especially around numerical computing.
    It was incorporated in the Jupyter ecosystem as a way to unify computational languages into a single
    development environment for data science tasks.
    Julia's \code{dataframes.jl} library was released in 2012, giving the same data manipulating
    features as the R dataframe object and Python's \code{pandas} library.

    Representing a dataframe object across multiple programming languages is a challenge because of
    how array objects are implement from language to language.
    The Apache Arrow project aims to create a centralized API for dataframes that can be used across multiple languages,
    allowing end users to use whatever tool best suits their needs,
    but using a unified and performant data structure
    \cite{ApacheArrow}.
    Groups like Ursa Labs were formed in 2018 (now Voltron Data in 2021) to help provide
    more support to the Apache Arrow project
    \cite{UrsaLabs, VoltronData}.

    The popularity of Javascript over the years (Rank \#7 on the TIOBE index in November 2021)
    stem from it connection with end user interactions with
    virtually all websites on the internet.
    It has become a tool for creating interactive figures (e.g., \code{D3.js}) for people on the internet.
    Since essentially every person who connects to the internet uses a web browser,
    WebAssembly was announced in 2015 as an open standard for any programming language
    to compile down for web applications to run in a browser
    \cite{WebAssembly}.

    Projects like Apache Arrow and WebAssembly (Wasm) are blurring the lines between tools needed for data science tasks.
    Apache Arrow is unifying data analysis by providing a single unified memory format to use data for analysis
    \cite{ApacheArrow}.
    The project has bindings for many programming languages in an effort to make data programming language independent.
    These tools are paving the way for end results to not rely on particular programming languages.

\end{document}
