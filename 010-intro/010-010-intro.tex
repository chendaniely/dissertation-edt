\documentclass[010-intro.tex]{subfiles}
\begin{document}

This dissertation describes the current state of education and pedagogy in the field of data science
and explores ways to improve data science education in a specific sub-field, biomedical sciences.
By critically appraising the field of data science education
and identifying gaps in learning and pedagogical research,
this work sought to determine research-backed solutions and approaches in data science education
as applied to the domain of biomedical sciences.
The main objective is to help future instructors better cater to novice data science learners
by providing the tools and methods needed to identify their learner's needs and create
more relevant learning materials for better engagement and long-term learning.

This discussion begins by orienting the reader with
a broad introduction of the past, current, and future of data science education.
This chapter describes the importance of delineating between
data science education and computer science education.
Additionally, this chapter describes the current gaps in
data science education and data literacy in the field of biomedical sciences.

The impact, need, and considerations of creating domain-specific data science materials
are included in this body of work.
This dissertation is organized in the way you would go about creating
a set of domain-specific data science learning materials.
First, by identify who the learners are and what are their relevant prior experiences with
data and programming;
Section \ref{se:intro-personas} introduces learner personas as a means to identify your learner audience
and the process of creating and using personas are described in Chapter \ref{ch:persona_validation}.
Once our audience is identified,
their background, relevant prior knowledge, perception of needs, and special considerations
are used to create a set of learning materials.
Section \ref{se:intro-teaching-best-practices-ds} introduces these learning materials,
and assesses how effective the materials are to meet the learning objectives.
The details of these learning materials and their effectiveness are described in Chapter \ref{ch:workshop}.
In section \ref{sse:intro-lesson-design} we look more closely into how learning works,
by looking at the formative assessment questions that are asked throughout the learning materials,
and how they play a role in the final summative assessment question that aims to summarize the learning objectives.
Details of this experiment are described in Chapter \ref{ch:assessment}.
Finally, Chapter \ref{ch:conclusion} summarizes the impact of this work in data science education,
and plans for the future in this domain of research and education.

\end{document}
