%     File: VTthesis_template.tex
%     Created: Thu Mar 24 11:00 AM 2016 EDT
%     Last Change: Thursday, December 19, 2019
%     Author: Alan M. Lattimer, VT
%     With modifications by Carrie Cross, Robert Browder, and LianTze Lim.
%
% This template is designed to operate with XeLaTeX.
%
% All elements in the Title, Abstract, and Keywords MUST be formatted as text and NOT as math.
%
%Further instructions for using this template are embedded in the document. Additionally, there are comments at the end of the file that give suggestions on writing your thesis.
%
%In addition to the standard formatting options, the following options are defined for the VTthesis class: proposal, prelim, doublespace, draft.

%\RequirePackage[final]{graphicx} % the do before documentclass and final makes the image show up in draft mode
%\documentclass[doublespace,draft,nopageskip]{VTthesis} % nopageskip - Removes arbitrary blank pages.
\documentclass[doublespace,final,nopageskip]{VTthesis} % nopageskip - Removes arbitrary blank pages.

\usepackage{amsmath, amssymb}

% remember to run makeindex for nomenclature / abbreviation list to show:
% https://tex.stackexchange.com/questions/27824/using-package-nomencl
\usepackage{nomencl}  % for nomenclature
\makenomenclature

%\usepackage{graphicx} % if you set final here you'll get option clash error
\usepackage{multirow} % for subfigure
\usepackage{hyperref}
\usepackage{subfiles}
\usepackage{csquotes} % for block quotes (displayquote)
\usepackage{listings} % inline "code"
\usepackage{enumitem} % label enumerate items

% https://tex.stackexchange.com/questions/2275/keeping-tables-figures-close-to-where-they-are-mentioned
\usepackage[section]{placeins} % place figures in the setion they're mentioned

% see https://tex.stackexchange.com/questions/3871/citing-a-range-of-papers-using-numeric-keys-as-in-citea-b-c-1-3
\usepackage[numbers,sort&compress]{natbib} %% this is set (changed from sort to sort%compress) in the VT template file

% https://stackoverflow.com/questions/29348517/fit-a-table-within-textwidth-in-latex
% fit table within textwidth
\usepackage{tabularx,booktabs}

% https://tex.stackexchange.com/questions/36030/how-to-make-a-single-word-look-as-some-code
\newcommand{\code}[1]{\lstinline{#1}}

\graphicspath{{./}{figs/}}

% Using the following header instead will create a draft copy of your thesis
%\documentclass[doublespace,draft]{VTthesis}

% Title of your thesis
\title{A Pedagogical Approach
       to Create and Assess
       Domain-Specific
       Data Science Learning Materials
       in the Biomedical Sciences}

% You should include 3-5 keywords, separated by commas
\keywords{data science, data science education, pedagogy, medical education, biomedical sciences}

% Your name, including middle initial(s)
\author{Daniel Y. Chen}

% Change this to your program, e.g. Physics, Civil Engineering, etc.
\program{Genetics, Bioinformatics, and Computational Biology}

% Change this to your degree, e.g. Master of Science, Master of Art, etc.
\degree{Doctor of Philosophy}

% This should be your defense date:
\submitdate{December 14, 2021}

% Committee members. Only have five readers and one chair available.
% Only use the ones you need and don't include the ones you don't need.
% You can also declare a Co-advisor. If you do, the principal and co-advisors
% will be listed as co-advisors on the title page.  Per the VT ETD standards,
% you should not include titles or educational qualifications such as PhD or Dr.
% You should, however, include middle initials if possible.
\principaladvisor{Anne M. Brown}
%\coadvisor{Vicente Esparza}
\firstreader{David M. Higdon}
\secondreader{Alexandra L. Hanlon}
\thirdreader{Stephanie N. Lewis}
%\fourthreader{Fourth Committee Member}
%\fifthreader{Fifth Committee Member}

% The dedication and acknowledgment pages are optional. Comment them out to remove them.
%\dedication{This is where you put your dedications.}
%\acknowledge{This is where you put your acknowledgments.}
\dedication{
    To my family, all the people who helped get me here, and all the great teachers I've had.
}

\acknowledge{
    All chapters included in this dissertation document were written by the candidate.
    Dr. Anne M. Brown served as primary research advisor and
    provided editorial comments and suggestions for improvement all manuscripts for publication.
    The specifics of author contributions are listed below by chapter.
    Each author's initials are used to specify which author contributed each part.
    Author names used are as follows:\\
    Daniel Y. Chen (DYC)\\
    Anne M. Brown (AMB)\\
    David M. Higdon (DMH)\\
    Alexandra L. Hanlon (ALH)\\
    Stephanie N. Lewis (SNL)
    \paragraph{Chapters 2-4}
    DYC wrote these papers with input on content and corrections prior to publication from AMB.
    DYC performed a majority of the survey design, adminstration, and analysis.
    DMH, ALH, and SNL provided input during the preliminary phase.
    ALH and her biostatistics consulting group helped with aditional data analysis questions.
    DYC and AMB collectively conceived the initial project idea and direction.
    DYC and AMB are responsible for submission of the completed manuscripts and response to reviewers.
    These chapters has not been published yet.
}

% The abstract is required.
\abstract{
    This dissertation explores
    creating a set of domain-specific learning materials for the biomedical sciences
    to meet the the educational gap in biomedical informatics,
    while also meeting the call for statisticians advocating for process improvements in other disciplines.
    Data science educational materials are plenty enough to become commodity.
    This provides the opportunity to create domain-specific learning materials
    to better motivate learning using real-world examples
    while also capturing intricacies of working with data in a specific domain.
    This dissertation shows how the use of persona methodologies can be combined
    with a backwards design approach of creating domain-specific learning materials.
    The work is divided into three (3) major steps:
    (1) create and validate a learner self-assessment survey that can identify learner personas by clustering.
    (2) combine the information from persona methodology with a backwards design approach
        using formative and summative assessments
        to curate, plan, and assess domain-specific data science
        workshop materials for short term and long term efficacy.
    (3) pilot and identify at how to manage real-time feedback within a
        data coding teaching session to drive better learner motivation and engagement.
    The key findings from this dissertation suggests using a structured framework to
    plan and curate learning materials is an effective way to identify key concepts
    in data science. However, just creating and teaching learning materials is not
    enough for long-term retention of knowledge. More effort for long-term lesson
    maintenance and long-term strategies for practice will help retain the concepts
    learned from live instruction. Finally, it is essential that we are careful
    and purposeful in our content creation as to not overwhelm learners and to
    integrate their needs into the materials as a primary focus.
    Overall, this contributes to the growing need for data science education
    in the biomedical sciences to train future clinicians use and work with data
    and improve patient outcomes.
}

% The general audience abstract is required. There are currently no word limits.
\abstractgenaud{
    Regardless of the field and domain you are in, we are all
    inundated with data. The more agency we can give individuals to work with data,
    the better equipped they will be to bring their own expertise to complex
    problems and work in multidisciplinary teams. There already exists a plethora
    of data science learning materials to help learners work with data; however,
    many are not domain-focused and can be overwhelming to new learners. By
    integrating in domain specificity to data science education, we hypothesize that
    we can help learners learn and retain knowledge by keeping them more engaged and
    motivated. This dissertation focuses on the domain of the biomedical sciences to
    use best practices on how to improve data science education and impact the
    field. Specifically, we explore how to address major gaps in data education in
    the biomedical field and create a set of domain-specific learning materials
    (e.g. workshops) for the biomedical sciences. We use best educational practices
    to curate these learning materials and assess how effective they are. This
    assessment was performed in three (3) major steps including: (1) identify who the
    learners are and what they already know in the context of using a programming
    language to work with data, (2) plan and curate a learning path for the
    learners and assessing materials created for short and long term effectiveness,
    and (3) pilot and identify at how to manage real-time feedback within a
    data coding teaching session to drive better learner motivation and engagement.
    The key findings from this dissertation suggests using a structured framework to
    plan and curate learning materials is an effective way to identify key concepts
    in data science. However, just creating the materials and teaching them is not
    enough for long-term retention of knowledge. More effort for long-term lesson
    maintenance and long-term strategies for practice will help retain the concepts
    learned from live instruction. Finally, it is essential that we are careful
    and purposeful in our content creation as to not overwhelm learners and to
    integrate their needs into the materials as a primary focus.
    Overall, this contributes to the growing need for data science education
    in the biomedical sciences to train future clinicians use and work with data
    and improve patient outcomes.
}

\pagenumbering{roman}

\begin{document}
% The following lines set up the front matter of your thesis or dissertation and are required to ensure proper formatting per the VT ETD standards.
  \frontmatter
  \maketitle
  \tableofcontents

% The list of figures and tables are now optional per the official ETD standards.  Unless you have a very good reason for removing them, you should leave these lists in the document. Comment them out to remove them.
    \listoffigures
    \listoftables
    \printnomenclature %Creates a list of abbreviations. Comment out to remove it.
    \subfile{005-frontmatter/050-abbreviations}

% The following sets up the document for the main part of the thesis or dissertation. Do not comment out or remove this line.
	\mainmatter

    %now go ahead and start writing your thesis
    \chapter{Introduction}
        \label{ch:introduction}
        \subfile{010-intro/010-intro}

    \chapter{Identification of Biomedical Data Science Learner Persons:
             Implications and Lessons Learned for Domain-Specific Data Science Curriculum}
        \label{ch:persona_validation}
        \subfile{020-persona_validation/020-persona_validation}

    \chapter{Assessing the Efficacy of Domain-Specific Data Science Curriculum in the Biomedical Sciences:
             How Learner Personas Can Guide Educational Needs in the Short-Term and Long-Term}
        \label{ch:workshop}
        \subfile{030-workshop/030-workshop}

    \chapter{Refining Feedback and Guidance in Data Science Workshops:
             Making Time for Formative and Summative Assessments Engages Students and Refines Lesson Content}
        \label{ch:assessment}
        \subfile{040-assessment/040-assessment}

    \chapter{Conclusion}
        \label{ch:conclusion}
        \subfile{050-conclusion/050-000-conclusion}


    % This is the standard bibtex file. Do not include the .bib extension in <bib_file_name>.
    % Uncomment the following lines to include your bibliography:
    \bibliographystyle{unsrt}
    \bibliography{bib}
    %\printbibliography

    % % This formats the chapter name to appendix to properly define the headers:
    % \appendix

    % % Add your appendices here. You must leave the appendices enclosed in the appendices environment in order for the table of contents to be correct.
    % \begin{appendices}
    %     \chapter{Participant Unique Identifier} \label{app:participant-id}
    %         \subfile{090-appendix/005-participant_id}
    %     \chapter{Persona Validation and Creation} \label{app:persona_validation_creation}
    %         \subfile{090-appendix/010-persona_validation_creation}
    %     \chapter{Workshop Efficacy} \label{app:workshop_efficacy}
    %         \subfile{090-appendix/020-workshop_pre_post_long}
    %     \chapter{Assessment Study} \label{app:exercises}
    %         \subfile{090-appendix/030-exercises}
    % \end{appendices}

\end{document}


%****************************************************************************
% Below are some general suggestions for writing your dissertation:
%
% 1. Label everything with a meaningful prefix so that you
%    can refer back to sections, tables, figures, equations, etc.
%    Usage \label{<prefix>:<label_name>} where some suggested
%    prefixes are:
%            ch: Chapter
%             se: Section
%             ss: Subsection
%             sss: Sub-subsection
%            app: Appendix
%             ase: Appendix section
%             tab: Tables
%             fig: Figures
%             sfig: Sub-figures
%             eq: Equations
%
% 2. The VTthesis class provides for natbib citations. You should upload
%     one or more *.bib bibtex files. Suppose you have two bib files: some_refs.bib and
%    other_refs.bib.  Then your bibliography line to include them
%    will be:
%      \bibliography{some_refs, other_refs}
%    where multiple files are separated by commas. In the body of
%    your work, you can cite your references using natbib citations.
%    Examples:
%      Citation                     Output
%      -------------------------------------------------------
%      \cite{doe_title_2016}        [18]
%      \citet{doe_title_2016}       Doe et al. [18]
%      \citet*{doe_title_2016}      Doe, Jones, and Smith [18]
%
%    For a complete list of options, see
%      https://www.ctan.org/pkg/natbib?lang=en
%
% 3. Here is a sample table. Notice that the caption is centered at the top. Also
%    notice that we use booktabs formatting. You should not use vertical lines
%    in your tables.
%
%                \begin{table}[htb]
%                    \centering
%                    \caption{Approximate computation times in hh:mm:ss for full order versus reduced order models.}
%                    \begin{tabular}{ccc}
%                        \toprule
%                        & \multicolumn{2}{c}{Computation Time}\\
%                        \cmidrule(r){2-3}
%                        $\overline{U}_{in}$ m/s & Full Model & ROM \\
%                        \midrule
%                        0.90 & 2:00:00 & 2:08:00\\
%                        0.88 & 2:00:00 & 0:00:03\\
%                        0.92 & 2:00:00 & 0:00:03\\
%                        \midrule
%                        Total & 6:00:00 & 2:08:06\\
%                        \bottomrule
%                    \end{tabular}
%                    \label{tab:time_rom}
%                \end{table}
%
% 4. Below are some sample figures. Notice the caption is centered below the
%    figure.
%    a. Single centered figure:
%                    \begin{figure}[htb]
%                        \centering
%                        \includegraphics[scale=0.5]{my_figure.eps}
%                        \caption{Average outlet velocity magnitude given an average
%                        input velocity magnitude of 0.88 m/s.}
%                        \label{fig:output_rom}
%                    \end{figure}
%    b. Two by two grid of figures with subcaptions
%                    \begin{figure}[htb]
%                        \centering
%                        \begin{subfigure}[h]{0.45\textwidth}
%                            \centering
%                            \includegraphics[scale=0.4]{figure_1_1.eps}
%                            \caption{Subcaption number one}
%                            \label{sfig:first_subfig}
%                        \end{subfigure}
%                        \begin{subfigure}[h]{0.45\textwidth}
%                            \centering
%                            \includegraphics[scale=0.4]{figure_1_2.png}
%                            \caption{Subcaption number two}
%                            \label{sfig:second_subfig}
%                        \end{subfigure}
%
%                        \begin{subfigure}[h]{0.45\textwidth}
%                            \centering
%                            \includegraphics[scale=0.4]{figure_2_1.pdf}
%                            \caption{Subcaption number three}
%                            \label{sfig:third_subfig}
%                        \end{subfigure}
%                        \begin{subfigure}[h]{0.45\textwidth}
%                            \centering
%                            \includegraphics[scale=0.4]{figure_2_2.eps}
%                            \caption{Subcaption number four}
%                            \label{sfig:fourth_subfig}
%                        \end{subfigure}
%                        \caption{Here is my main caption describing the relationship between the 4 subimages}
%                        \label{fig:main_figure}
%                    \end{figure}
%
%----------------------------------------------------------------------------
%
% The following is a list of definitions and packages provided by VTthesis:
%
% A. The following packages are provided by the VTthesis class:
%      amsmath, amsthm, amssymb, enumerate, natbib, hyperref, graphicx,
%      tikz (with shapes and arrows libraries), caption, subcaption,
%      listings, verbatim
%
% B. The following theorem environments are defined by VTthesis:
%      theorem, proposition, lemma, corollary, conjecture
%
% C. The following definition environments are defined by VTthesis:
%      definition, example, remark, algorithm
%
%----------------------------------------------------------------------------
%
%  I hope this template file and the VTthesis class will keep you from having
%  to worry about the formatting and allow you to focus on the actual writing.
%  Good luck, and happy writing.
%    Alan Lattimer, VT, 2016
%
%****************************************************************************
