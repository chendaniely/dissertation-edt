\documentclass[030-workshop.tex]{subfiles}

\begin{document}

The demand for clinical informatics training is outpacing the supply and opportunities for training.
Even though there is an abundance of data science training materials,
there are not enough domain-specific data science materials in the biomedical sciences.
In an effort to plan on developing these domain-specific materials,
a previous first identified learner personas which included background information,
relevant prior knowledge, perception of needs, and special considerations of learners.
These personas were used to create a set of learning materials using backward design
to create a set of open access relevant domain-specific learning materials for the biomedical sciences.

This study looks at the efficacy of these learning materials using a set of
cross-sectional longitudinal surveys that track confidence in meeting learning objectives
and answering a summative assessment question.
200 total workshop participants participated in 67 pre-workshop surveys,
43 post-workshop surveys, and 11 long-term workshop surveys.
The study sees an improvement in learner's confidence in meeting learning objectives
post-workshop, but in the long-term survey (at least 4 months out),
confidence in meeting learning objectives and confidence to complete a summative assessment question
that covered data loading, subsetting, saving, tidying, and model fitting were back to pre-workshop levels.
This suggests that lesson materials may have an effect to learners in the short term,
but more resources need to be created and provided in the long-term
for learners.

The authors recommend that time and effort into creating domain-specific data science materials from scratch
can be better served by creating more case-study and data examples to serve learners in the long term.
Introductory materials can be created by remixing existing bodies of work,
and efforts into creating relevant real-world examples can be used for formative assessment questions
during a lesson.
This curation of domain-specific exercises and case studies will be easier to maintain for
the original authors as well as levering a broader community of practice for lesson
content maintenance.

\end{document}