\documentclass[030-workshop.tex]{subfiles}

\begin{document}

\subsection{Post-Workshop Survey Questions}
\label{sse:postworkshop-survey-questions}

\subsubsection{Demographics}

    \paragraph{Q2.2}

        Please create a unique identifier.
        This unique identifier will be used for long-term assessment but keep your personal information anonymous.

        To create an identifier type in:
        Number of siblings (as numeric) +
        First two letters of the city you were born in (lowercase) +
        First three letters of your current street (lowercase).

        E.g., (Sherlock Homes has \textbf{1} brother,
            was born in \textbf{Po}rsmouth,
            and lives on \textbf{Bac}ker Street - \textbf{1pobac})

    \paragraph{Q2.3}

        Please select the first date of your workshop

        \begin{itemize}
            \item Tuesday, June 29, 2021  (7)
            \item Tuesday, May 18, 2021  (6)
            \item Tuesday, February 2, 2021  (5)
            \item Wednesday, December 9, 2020  (4)
            \item Tuesday, October 20, 2020  (1)
            \item I went through the online materials on my own  (2)
            \item Other  (3) \_\_\_
        \end{itemize}

    \paragraph{Q2.4}

    What is your current occupation/career stage (select all that apply).

    \begin{itemize}
        \item DO/MD  (1)
        \item DVM  (12)
        \item RN/PA  (2)
        \item PhD  (13)
        \item Academic  (3)
        \item Analyst  (4)
        \item Student (Masters e.g., MPH)  (5)
        \item Student (MD/DO)  (6)
        \item Student (Nurse, PA)  (7)
        \item Student (Graduate)  (8)
        \item Student (Undergraduate)  (9)
        \item iTHRIV Scholar  (11)
        \item Other, please describe  (10)
    \end{itemize}

\subsubsection{Workshop Environment}

    \paragraph{Q3.1}

        Please rate your level of agreement with the following statements:

        \begin{itemize}
            \item Strongly Disagree (1)
            \item Disagree (2)
            \item Somewhat Disagree (3)
            \item Neither Agree nor Disagree (4)
            \item Somewhat Agree (5)
            \item Agree (6)
            \item Strongly Agree (7)
        \end{itemize}

        \begin{itemize}
            \item I felt comfortable learning in this environment (1)
            \item I can immediately apply what I learned (2)
            \item I was able to get clear answers to my questions from the instructors (3)
            \item The instructors were enthusiastic about the workshop (4)
            \item I felt comfortable interacting with the instructors (5)
            \item The instructors were knowledgeable about the material being taught (6)
        \end{itemize}

    \paragraph{Q3.2}

        Do you have accessibility requirements?

        \begin{itemize}
            \item No  (1)
            \item Yes  (2)
        \end{itemize}

    \paragraph{Q3.3}

        Where there any accessibility issues that affected your ability to participate in this workshop?

        \begin{itemize}
            \item No  (1)
            \item Yes  (2)
            \item Not applicable  (3)
        \end{itemize}


    \paragraph{Q3.4}

        Please describe what the accessibility issues were.

\subsubsection{Workshop Framing and Motivation}

    \paragraph{Q4.1}

        Please rate your level of agreement with the following statements:

        \begin{itemize}
            \item Strongly Disagree (1)
            \item Disagree (2)
            \item Somewhat Disagree (3)
            \item Neither Agree nor Disagree (4)
            \item Somewhat Agree (5)
            \item Agree (6)
            \item Strongly Agree (7)
        \end{itemize}

        \begin{itemize}
            \item I believe having access to the original, raw data is important to be able to repeat an analysis. (1)
            \item I can write a small program, script, or macro to address a problem in my own work. (2)
            \item I know how to search for answers to my technical questions online. (3)
            \item While working on a programming project, if I got stuck, I can find ways of overcoming the problem. (4)
            \item I am confident in my ability to make use of programming software to work with data. (5)
            \item Using a programming language (like R or Python) can make my analyses easier to reproduce. (6)
            \item Using a programming language (like R or Python) can make me more efficient at working with data. (7)
        \end{itemize}

    \paragraph{Q4.2}

        Please rate your level of agreement with the following statements:

        \begin{itemize}
            \item Strongly Disagree (1)
            \item Disagree (2)
            \item Somewhat Disagree (3)
            \item Neither Agree nor Disagree (4)
            \item Somewhat Agree (5)
            \item Agree (6)
            \item Strongly Agree (7)
        \end{itemize}

        \begin{itemize}
            \item Name the features of a tidy/clean dataset (1)
            \item Transform data for analysis (2)
            \item Identify when spreadsheets are useful (3)
            \item Assess when a task should not be done in a spreadsheet software (4)
            \item Break down data processing into smaller individual (and more manageable) steps (5)
            \item Construct a plot and table for exploratory data analysis (6)
            \item Build a data processing pipeline that can be used in multiple programs (7)
            \item Calculate, interpret, and communicate an appropriate statistical analysis of the data (8)
        \end{itemize}

\subsubsection{Summative assessment}

    \paragraph{Q5.1}

    Cytomegalovirus (CMV) is a common virus that normally does not cause any
    problems in the body. However, it can be of concern for those who are
    pregnant or immunocompromised.

    Suppose you have the following Cytomegalovirus dataset [1] of CMV reactivation
    among patients after Allogenetic Hematopoietic Stem Cell Transplant (HSCT) in an
    excel sheet (first 10 rows shown below):

    It contains a patient's: ID age priror.radiation: whether or not patient had
    prior radiation treatment (0 = no, 1 = yes) aKIRs: Number of donor activating
    killer immunoglobulin-link receptors donor\_negative: the recipient's CMV status
    when the donor was CMV negative donor\_positive: the recipient's CMV status when
    the donor was CMV positive

    It is believed that the donor activating KIR genotype is a contributing factor
    for CMV reactivation after myeloablastive allogenetic HSCT. You want to do some
    data analysis to see what variables are associated with CMV reactivation.

    Assuming this is the version of the data you need for the tidying, plotting, and
    modeling:

    How would you rate your ability to accomplish the following tasks:

    [1]: Sobecks et al. ``Cytomegalovirus Reactivation After Matched Sibling
    DonorReduced-Intensity Conditioning Allogeneic HematopoieticStem Cell Transplant
    Correlates With Donor KillerImmunoglobulin-like Receptor Genotype''. Exp Clin
    Transplant2011; 1: 7-13.

    \begin{itemize}
        \item I wouldn't know where to start (4)
        \item I could struggle through, but not confident I could do it (5)
        \item I could struggle through by trial and error with a lot of web searches (6)
        \item I could do it quickly with little or no use of external help (7)
    \end{itemize}

    \begin{itemize}
        \item Load the excel sheet into R (1)
        \item Filter the data for individuals over the age of 65 (in R) (2)
        \item Save filtered dataset (in R) as an Excel file to send to a colleague (6)
        \item Tidy the dataset (in R) so we have a donor CMV status and a patient CMV status in separate columns (3)
        \item Plot a histogram (in R) of the age distribution of our data (4)
        \item Fit a model (e.g., logistic regression) to see which variables are associated with patient CMV reactivation  (in R) (5)
    \end{itemize}

\subsubsection{Workshop Content}

    \paragraph{Q6.1}

        Were there any topics you wish were covered?

    \paragraph{Q6.2}

        What topic would you take out of the workshop to make room for the topics mentioned above?

    \paragraph{Q6.3}

        In general, how would you prefer to have the workshop content (4 - 5 hours) taught?

    \begin{itemize}
        \item 1 day 4-5 hour workshop on the weekday  (1)
        \item 1 day 4-5 hour workshop on the weekend  (2)
        \item 2 days about 2-3 hours each on the weekday  (3)
        \item 2 days about 2-3 hours each on the weekday  (4)
        \item Multiple days in a row about 1 hour each day  (5)
        \item Multiple days spread across multiple weeks on the weekdays  (6)
        \item Multiple days spread across multiple weeks on the weekends  (7)
        \item Not applicable  (8)
    \end{itemize}

\subsubsection{Open Feedback}

    \paragraph{Q7.1}

        Please provide an example of how an instructor or helper affected your learning experience.

    \paragraph{Q7.2}

        What is something you liked about the workshop?

    \paragraph{Q7.3}

        What is something you \textbf{did not} like about the workshop?

\end{document}