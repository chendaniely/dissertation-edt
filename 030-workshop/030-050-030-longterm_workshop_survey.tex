\documentclass[030-workshop.tex]{subfiles}

\begin{document}

\subsection{Long-Term Workshop Survey Questions}
\label{sse:longtermworkshop-survey-questions}

\subsubsection{Demographics}

    \paragraph{Q2.2}

    Please create a unique identifier.
    This unique identifier will be used for long-term assessment but keep your personal information anonymous.

    To create an identifier type in:
    Number of siblings (as numeric) +
    First two letters of the city you were born in (lowercase) +
    First three letters of your current street (lowercase).

    E.g., (Sherlock Homes has \textbf{1} brother,
        was born in \textbf{Po}rsmouth,
        and lives on \textbf{Bac}ker Street - \textbf{1pobac})

    \paragraph{Q2.3}

    Please select the first date of your workshop

    \begin{itemize}
        \item Tuesday, June 29, 2021  (7)
        \item Tuesday, May 18, 2021  (6)
        \item Tuesday, February 2, 2021  (5)
        \item Wednesday, December 9, 2020  (4)
        \item Tuesday, October 20, 2020  (1)
        \item I went through the online materials on my own  (2)
        \item Other  (3) \_\_\_
    \end{itemize}

    \paragraph{Q2.4}

    What is your current occupation/career stage (select all that apply).

    \begin{itemize}
        \item DO/MD  (1)
        \item DVM  (12)
        \item RN/PA  (2)
        \item PhD  (13)
        \item Academic  (3)
        \item Analyst  (4)
        \item Student (Masters e.g., MPH)  (5)
        \item Student (MD/DO)  (6)
        \item Student (Nurse, PA)  (7)
        \item Student (Graduate)  (8)
        \item Student (Undergraduate)  (9)
        \item iTHRIV Scholar  (11)
        \item Other, please describe  (10)
    \end{itemize}

\subsubsection{Behaviors and Confidence}

    \paragraph{Q3.1}

        Which of the following behaviors have you adopted as a result of completing the workshop / going through the materials.

        \begin{itemize}
            \item Improving data management and project organization  (1)
            \item Developing a data management and analysis plan  (2)
            \item Transforming step-by-step workflows into scripts  (3)
            \item Using programming languages like R or Python to automate repetitive tasks  (4)
            \item Reusing code  (5)
            \item Sharing code or data publicly  (6)
            \item None  (12)
            \item Other  (13) \_\_
        \end{itemize}

    \paragraph{Q3.2}

        Before the workshop, how often did you use programming languages?

        \begin{itemize}
            \item I had not been using tools like these  (1)
            \item Less than once a per half-year  (2)
            \item Several times per half-year  (3)
            \item Monthly  (4)
            \item Weekly  (5)
            \item Daily  (6)
        \end{itemize}

    \paragraph{Q3.3}

        Since taking the workshop, how often did you use programming languages?

        \begin{itemize}
            \item I had not been using tools like these  (1)
            \item Less than once a in the last 6 months  (2)
            \item Several times in the last 6 months  (3)
            \item Monthly  (4)
            \item Weekly  (5)
            \item Daily  (6)
        \end{itemize}

    \paragraph{Q3.4}

        How would you rate your change in confidence in the tools that were covered during your workshop compared to before the workshop?
        \begin{itemize}
            \item I'm more confident now  (1)
            \item I'm equally confident now  (2)
            \item I'm less confident now  (3)
        \end{itemize}

\subsubsection{Workshop Framing and Motivation}

    \paragraph{Q4.1}

    Please rate your level of agreement with the following statements:

    \begin{itemize}
        \item Strongly Disagree (1)
        \item Disagree (2)
        \item Somewhat Disagree (3)
        \item Neither Agree nor Disagree (4)
        \item Somewhat Agree (5)
        \item Agree (6)
        \item Strongly Agree (7)
    \end{itemize}

    \begin{itemize}
        \item I believe having access to the original, raw data is important to be able to repeat an analysis. (1)
        \item I can write a small program, script, or macro to address a problem in my own work. (2)
        \item I know how to search for answers to my technical questions online. (3)
        \item While working on a programming project, if I got stuck, I can find ways of overcoming the problem. (4)
        \item I am confident in my ability to make use of programming software to work with data. (5)
        \item Using a programming language (like R or Python) can make my analyses easier to reproduce. (6)
        \item Using a programming language (like R or Python) can make me more efficient at working with data. (7)
    \end{itemize}

    \paragraph{Q4.2}

    Please rate your level of agreement with the following statements:

    \begin{itemize}
        \item Strongly Disagree (1)
        \item Disagree (2)
        \item Somewhat Disagree (3)
        \item Neither Agree nor Disagree (4)
        \item Somewhat Agree (5)
        \item Agree (6)
        \item Strongly Agree (7)
    \end{itemize}

    \begin{itemize}
        \item Name the features of a tidy/clean dataset (1)
        \item Transform data for analysis (2)
        \item Identify when spreadsheets are useful (3)
        \item Assess when a task should not be done in a spreadsheet software (4)
        \item Break down data processing into smaller individual (and more manageable) steps (5)
        \item Construct a plot and table for exploratory data analysis (6)
        \item Build a data processing pipeline that can be used in multiple programs (7)
        \item Calculate, interpret, and communicate an appropriate statistical analysis of the data (8)
    \end{itemize}

\subsubsection{Summative assessment}

    \paragraph{Q5.1}

    Cytomegalovirus (CMV) is a common virus that normally does not cause any
    problems in the body. However, it can be of concern for those who are
    pregnant or immunocompromised.

    Suppose you have the following Cytomegalovirus dataset [1] of CMV reactivation
    among patients after Allogenetic Hematopoietic Stem Cell Transplant (HSCT) in an
    excel sheet (first 10 rows shown below):

    It contains a patient's: ID age priror.radiation: whether or not patient had
    prior radiation treatment (0 = no, 1 = yes) aKIRs: Number of donor activating
    killer immunoglobulin-link receptors donor\_negative: the recipient's CMV status
    when the donor was CMV negative donor\_positive: the recipient's CMV status when
    the donor was CMV positive

    It is believed that the donor activating KIR genotype is a contributing factor
    for CMV reactivation after myeloablastive allogenetic HSCT. You want to do some
    data analysis to see what variables are associated with CMV reactivation.

    Assuming this is the version of the data you need for the tidying, plotting, and
    modeling:

    How would you rate your ability to accomplish the following tasks:

    [1]: Sobecks et al. “Cytomegalovirus Reactivation After Matched Sibling
    DonorReduced-Intensity Conditioning Allogeneic HematopoieticStem Cell Transplant
    Correlates With Donor KillerImmunoglobulin-like Receptor Genotype”. Exp Clin
    Transplant2011; 1: 7-13.

    \begin{itemize}
        \item I wouldn’t know where to start (4)
        \item I could struggle through, but not confident I could do it (5)
        \item I could struggle through by trial and error with a lot of web searches (6)
        \item I could do it quickly with little or no use of external help (7)
    \end{itemize}

    \begin{itemize}
        \item Load the excel sheet into R (1)
        \item Filter the data for individuals over the age of 65 (in R) (2)
        \item Save filtered dataset (in R) as an Excel file to send to a colleague (6)
        \item Tidy the dataset (in R) so we have a donor CMV status and a patient CMV status in separate columns (3)
        \item Plot a histogram (in R) of the age distribution of our data (4)
        \item Fit a model (e.g., logistic regression) to see which variables are associated with patient CMV reactivation  (in R) (5)
    \end{itemize}

\subsubsection{Impact}

    \paragraph{Q6.1}

        The statements below reflect ways in which completing the workshop may have impacted you.
        Please indicate your level of agreement with these statements.

        \begin{itemize}
            \item Strongly disagree (1)
            \item Disagree (2)
            \item Neutral (3)
            \item Agree (4)
            \item Strongly agree (5)
        \end{itemize}

        \begin{itemize}
            \item I have used skills I learned at the workshop to advance my career. (1)
            \item I have been motivated to seek more knowledge about the tools I learned at the workshop. (2)
            \item I have made my analysis more reproducible as a result of completing the workshop. (3)
            \item I have improved my coding practices as a result of completing the Workshop (4)
            \item My research productivity has improved as a result of completing the workshop (5)
            \item I have gained confidence in working with data as a result of completing the workshop. (6)
        \end{itemize}

    \paragraph{Q6.2}

        Did you go back to the online materials after the workshop?

        \begin{itemize}
            \item I went back to the code I wrote for reference  (1)
            \item I went back to the code the instructor posted for reference  (2)
            \item I went back to the video recording for the workshop  (3)
            \item I went back to the online written materials for reference  (4)
            \item I did not go back to any of the workshop materials  (5)
            \item Other  (6) \_\_
        \end{itemize}

    \paragraph{Q6.3}

        Why did you not go back to a particular workshop resource?

    \paragraph{Q6.4}

        Please tell us the most important way you were impacted as a result of the workshop.

    \paragraph{Q6.5}

        Please provide any outcomes as a result of attending this workshop.

    \paragraph{Q6.6}

        If you would like to make additional comments about the workshop experience,
        or ways you've used the tools you learned in the workshop please comment
        below.

\end{document}