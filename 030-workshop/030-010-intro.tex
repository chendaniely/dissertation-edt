\documentclass[030-workshop.tex]{subfiles}

\begin{document}

    There is a growing need for clinical infromatics training,
    but the demand is exceeding the opportunties to learn the relevant these skills to improve patient care.
    Some of demand can be met with more marketing and publically promoting existing resouces.
    However, increasing quantitiy, quality, and publicity, would all need to be incorporated
    in meeting the growing demand.

    There is a plethora of introductory data science materials,
    as the field continues to grow and curriculumns are adapted from K-12 to higher education,
    the base of knowledge in the general population will increase,
    however, what is currently lacking is providing domain specific data science curricums
    for working professionals who have more immediate data science needs, e.g., those who work in medicine.

    ``How Leraning Works'' provide seven (7) principles of learning:
    (1) Students' prior knowledge can help or hinder learning.
    (2) How students organize knowledge influences how they learn and apply what they know.
    (3) Students' motivation determines, directs, and sustains what they do to learn.
    (4) To develop mastery, students must acquire components skills, practive integrating them, and know when to apply what they have learned.
    (5) Goal-directed practice coupled with targed feedback enhances the quality of students' learning.
    (6) Students' current level of development interacts with the social, emotional, and intellectual climate of the course to impact learning.
    (7) To become self-directed learners, students must learn to monitor and adjust their approaches to learning.
    Getting a sense of what learners know (principles 1 and 2) is crucial in targeting
    the correct learning content to them.
    if we narrow the scope of potential learners to a specific domain,
    it will be easier to provide better and more applicable teaching examples to learners
    to help movibate them to learn (principle 3).
    Using persona methodologies to create learner personas help give more concrete examples
    of what learners know and what their needs are.
    This can also help identify any special needs to make the learning environment more condusive to learning (principle 6).

    Since courses take place over a fixed period of time,
    spurring internal motivation to continue practicing and learning (principle 5) is a challenge when putting together curriculum.
    Applying and practicing component skills is a core component into developing mastery (principle 4).
    Metacognition (principle 7) is a higher order skill and requires of self-reflection on the learner's end
    to adpat how they think and approach a problem.
    This is a skill that is often overlooked and neglected in many courses.
    Since many other steps in the learning process can happen after the course time,
    a way to scale eduation is to identify and leverage learning communities
    and build a community of practice. % TODO many points in overcome education challenges fit into here

    Creating domain specific materials help learners by showing examples that are more relevant.
    This helps with internal factors for motivation, and aids in creating learning feedback loops.
    These are all components on creating self-directed learnienrs.
    Our previous work used a learner self-assessment survey (i.e., persona survey) to create learner personas:
    Alex Academic, Clare Clinician, and Samir Student were the main groups of learners.
    Each group had varying amounts of
    programing experience, data programing experience, and
    confidence to search for and understand technical help on the internet.
    When we looked at overall responses to Excel useage and data literacy questions,
    we found that the vast majority of responses are famaliar with basic data analysis features to understand their data,
    e.g., able to calculate agrregate descriptive statistics,
    pivot tables,
    formulas,
    and plotting. % TODO put figures in apendix
    Most of the respondants also have an Excel-centric data workflow,
    and only a fraction interact with data programatically. % TODO put figure in appendix or something.
    These responses feed into how familar respondants were to data literacy jargon, i.e., ``long'' and ``wide'' data,
    and ``dummy variable'' and ``one-hot encoding'' of variables.

    The effectiveness of a curriumum can be operationalized by creating a concrete set of
    measureable learning objectives (LOs).
    Bloom's taxonomy serve as a useful model to create LOs.
    However, creating the learing objectives is not enough.
    The Computing Curricula guidelines have moved away from knowledge-based teaching to compentency-based teaching.
    Knowledge-based teaching aims to identify learners' current amount of knowledge,
    and starts to expand on the existing knowledge base.
    However, this approach has lead to gaps in applicable skills between the classroom and workforce.
    Since, a more compentency-based approach has been adapted,
    combining knowledge, skill, and dispositioin (i.e., what, how, and why) in order
    to have a deeper understanding and applicaiable set of skills.

    The learner personas helped identify ``tidy data'' as a core component in the data science process,
    and all of the LOs stemmed from tidy data principles.
    Since our learners primiarly use Excel,
    we felt that diving directly into loading data into a programming language would be too jarring of a transision.
    We followed the Data Carpentry curriculumns that put a spreadsheet lesson before going into the programming aspect.
    The spreadsheet lesson introduces tidy data principles without explitly naming them,
    rather it talks about why some datasets are more difficult to work with,
    and has learners curate a hypothetical pharmakonetics (PK) study where different ways of entering data are discussed.
    The data is then used to import into a programming language to help with example contunity.

    The learning objectives were used to assess learning material and workshop presentation effacicy,
    and provided longidutinal data with a pre-workshop, post-workshop, and long-term survey.
    These results could be compared to look at baseline data science and data literacy competencies,
    and how they change after the workshop,
    and how much information is retained long-term.
    One key component after the workshop was to provide other learning communities for learners to join
    that can help continue the learning process.
    By providing resouces to ask questions, other communities of practice,
    and other learners in the same biomedical domain,
    we hope it helps motivate learners to continue learning and work towards being self-directed learners.

\end{document}
