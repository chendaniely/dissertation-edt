\documentclass[030-workshop.tex]{subfiles}

\begin{document}

\subsection{Pre-Workshop Survey Questions}
\label{sse:preworkshop-survey-questions}

The surveys can be downloaded from the GitHub URL that holds the IRB proposal for the study:
\url{https://github.com/chendaniely/dissertation-irb/tree/master/irb-20-537-data\_science_workshops/survey}

\subsubsection{Demographics}

\paragraph{Q2.2}

    Please create a unique identifier.
    This unique identifier will be used for long-term assessment but keep your personal information anonymous.

    To create an identifier type in:
    Number of siblings (as numeric) +
    First two letters of the city you were born in (lowercase) +
    First three letters of your current street (lowercase).

    E.g., (Sherlock Homes has \textbf{1} brother,
        was born in \textbf{Po}rsmouth,
        and lives on \textbf{Bac}ker Street - \textbf{1pobac})

\paragraph{Q2.3}

    Please select the first date of your workshop

    \begin{itemize}
        \item Monday, September 20, 2021: Virtual  (9)
        \item Monday, September 20, 2021: In-Person  (8)
        \item Wednesday, September 22, 2021: Virtual  (10)
        \item Wednesday, September 22, 2021: In-Person  (11)
        \item Tuesday, June 29, 2021  (7)
        \item Monday, May 17, 2021  (6)
        \item Tuesday, February 2, 2021  (5)
        \item Wednesday, December 9, 2020  (4)
        \item Tuesday, October 20, 2020  (1)
        \item I went through the online materials on my own  (2)
    \end{itemize}

\paragraph{Q2.5}

    What is your current occupation/career stage (select all that apply).

    \begin{itemize}
        \item DO/MD  (1)
        \item DVM  (12)
        \item RN/PA  (2)
        \item PhD  (13)
        \item Academic  (3)
        \item Analyst  (4)
        \item Student (Masters e.g., MPH)  (5)
        \item Student (MD/DO)  (6)
        \item Student (Nurse, PA)  (7)
        \item Student (Graduate)  (8)
        \item Student (Undergraduate)  (9)
        \item iTHRIV Scholar  (11)
        \item Other, please describe  (10)
    \end{itemize}

\paragraph{Q2.6}

     What operating system will be on the computer you are using at the workshop or to participate in the online materials?

    \begin{itemize}
        \item Windows  (1)
        \item macOS  (2)
        \item Linux  (3)
        \item Not sure  (4)
    \end{itemize}

\subsubsection{Persona}

\paragraph{Q3.1}

Which of the below personas do you most identify with? Be less concerned about the actual occupation, and more with what relates to your skill and workshop needs.

More detailed descriptions of Alex Academic, Clare Clinician, Patricia Programmer, and Samir Student can be found here: https://ds4biomed.tech/who-is-this-book-for.html


\textbf{Alex Academic}

 Alex performs their research using a combination of Excel spreadsheets and specialized software,
 but is switching to R or Python (which they taught themself during a sabbatical).
 They have never taken a formal programming course, and suffers from impostor syndrome in discussions about programming.
 Alex would like to learn more about how programming can help their research and keep up with the tools their students are learning in class.

 Alex needs workshops (so they can allocate focused time) and how-to guides (for research).
 They would like ready-to-use lesson material that could be remixed for their students and some orientation material to demystify jargon (what is "tidy data"?).
 Alex also wants to be able to use the same tools in their research as in their teaching to amortize learning costs and stay in practice.

\textbf{Clare Clinician}

 Clare keeps up with medical research, but has little to no experience in doing medical research.
 They use Excel for non-data related tasks (e.g., making lists), or manually inputting patient data into spreadsheets for chart reviews.
 Wants to be able to collect and manage data as well as learn about the process behind data analysis to perform their own analysis and study one day.

Clare wants self-paced tutorials with practice exercises, plus forums where they can ask for help.
 They also need short overviews to orient them and introductory tutorials that include videos or animated GIFs showing exactly how to drive the tools,
 and that use datasets they can relate to.
 Clare wishes they had a community of other people in the medical field who are interested in learning how to do data work so they can learn and ask questions.

\textbf{Patricia Programmer}

 Patricia regularly connects to a remote server to do their work.
 They write SQL statements to pull data out of Epic and processes the data in both Python and R to generate reports and dashboards for their team and management.
 Patricia writes data pipelines for all their work either by combining shell scripts or build scripts.

 Patricia wants how-to guides and reference material for their day-to-day work
 and short, intensive online training for very specific topics.
 Because they often jump around between various tools, Patricia wants a way to quickly review topics before starting a new project.

\textbf{Samir Student}

 Samir is fairly proficient in Excel and does works with spreadsheets regularly and
 knows how to load up Excel spreadsheets into R and do basic data processing and analysis.
 However, they do not have that much practice outside of a classroom homework and project setting,
 and spends a lot of their time on StackOverflow copying and pasting code so they don't consider themselves a "real programmer".
 They have no problem getting their work done, but usually involves a lot of googling to eventually get the solution.

 Samir wants a formal workshop and reference materials that can be used to build a good foundation of the programming skills they were never taught.
 They want a better understanding of the terminology and jargon used in data science so they have the vocabulary to
 search for and understand solutions posted online. They are also looking for a community to help in their growth as a student in this domain.

\begin{itemize}
    \item Alex Academic  (1)
    \item Clare Clinician  (2)
    \item Patricia Programmer  (3)
    \item Samir Student  (4)
\end{itemize}

\subsubsection{Prior and background knowledge}

    \paragraph{Q4.1}

    How familiar are you with interactive programming languages like Python or R?

    \begin{itemize}
        \item I do not know what those are  (1)
        \item I have heard of them but have never used them before  (2)
        \item I have installed it, but have only done simple examples with them  (3)
        \item I have written a small program with them before  (4)
        \item I use it to automate certain repetitive tasks  (5)
        \item I have small side projects that I program in it  (6)
        \item I program in them for work  (7)
    \end{itemize}

    \paragraph{Q4.2}

    Are you familiar with the term ``tidy data''?

    \begin{itemize}
        \item I have never heard of the term  (1)
        \item I have heard of it but don’t remember what it is.  (2)
        \item I have some idea of what it is, but am not too clear  (3)
        \item I know what it is and could explain what it pertains to  (4)
    \end{itemize}

    \paragraph{Q4.3}

    If you were given a dataset containing an individual's smoking status
    (binary variable) and whether or not they have hypertension (binary variable),
    would you know how to conduct a statistical analysis to see
    if smoking has an increased relative risk or odds of hypertension? Any
    type of model will suffice.

    \begin{itemize}
        \item I wouldn't know where to start  (1)
        \item I could struggle through, but not confident I could do it  (4)
        \item I could struggle through by trial and error with a lot of web searches  (2)
        \item I could do it quickly with little or no use of external help  (3)
    \end{itemize}

\subsubsection{Workshop Framing and Motivation}

    \paragraph{Q5.1}

        Why are you participating in this workshop? Please check all that apply.

        \begin{itemize}
            \item To learn new skills  (1)
            \item To refresh or review my skills  (2)
            \item To learn skills that I can apply to my current work  (3)
            \item To learn skills that I can apply to my work in the future  (4)
            \item To learn skills that will help me get a job or a promotion  (5)
            \item As a requirement for my program or current position  (6)
        \end{itemize}

    \paragraph{Q5.2}

        Please rate your level of agreement with the following statements:

        \begin{itemize}
            \item Strongly Disagree (1)
            \item Disagree (2)
            \item Somewhat Disagree (3)
            \item Neither Agree nor Disagree (4)
            \item Somewhat Agree (5)
            \item Agree (6)
            \item Strongly Agree (7)
        \end{itemize}

        \begin{itemize}
            \item I believe having access to the original, raw data is important to be able to repeat an analysis. (1)
            \item I can write a small program, script, or macro to address a problem in my own work. (2)
            \item I know how to search for answers to my technical questions online. (3)
            \item While working on a programming project, if I got stuck, I can find ways of overcoming the problem. (4)
            \item I am confident in my ability to make use of programming software to work with data. (5)
            \item Using a programming language (like R or Python) can make my analyses easier to reproduce. (6)
            \item Using a programming language (like R or Python) can make me more efficient at working with data. (7)
        \end{itemize}

    \paragraph{Q5.3}

        Please rate your level of agreement with the following statements:

        \begin{itemize}
            \item Strongly Disagree (1)
            \item Disagree (2)
            \item Somewhat Disagree (3)
            \item Neither Agree nor Disagree (4)
            \item Somewhat Agree (5)
            \item Agree (6)
            \item Strongly Agree (7)
        \end{itemize}

        \begin{itemize}
            \item Name the features of a tidy/clean dataset (1)
            \item Transform data for analysis (2)
            \item Identify when spreadsheets are useful (3)
            \item Assess when a task should not be done in a spreadsheet software (4)
            \item Break down data processing into smaller individual (and more manageable) steps (5)
            \item Construct a plot and table for exploratory data analysis (6)
            \item Build a data processing pipeline that can be used in multiple programs (7)
            \item Calculate, interpret, and communicate an appropriate statistical analysis of the data (8)
        \end{itemize}

    \paragraph{Q5.4}

        Please share what you most hope to learn from participating in this workshop and/or workshop series.

    \paragraph{Q5.5}

        What do you want to know or be able to do after this workshop (or series of sessions) that you don't know or can't do right now?

\end{document}
