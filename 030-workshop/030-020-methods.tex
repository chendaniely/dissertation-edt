\documentclass[030-workshop.tex]{subfiles}

\begin{document}


  \subsection{Creating Learning Objectives}

    The initial persona survey showed that many of the potential learners who do not
    have existing data programming experience use Excel for basic calculations.
    We used the information from the persona survey to create a roadmap of topics
    to get learners from zero programming experience to fit a logistic regression model.
    These learning objectives were used to write the lesson materials
    and used as a Likert scale question in the pre-workshop and post-workshop surveys.

  \subsection{Workshop Materials: ds4biomed}

    Results from the survey pre-workshop learner self-assessment survey (i.e., persona survey)
    were used to create the workshop materials.
    The survey results from the first round of respondents were used to create the initial
    learning materials.
    Prioritizing data literacy concepts was the main goal of the learning materials.
    We used the framing from Data Carpentry lessons to frame the first spreadsheet chapter.
    The authors felt that this provided the most tangible connection from existing knowledge
    working in spreadsheets, mainly Excel,
    where common ``tidy data'' issues are discussed and why learners may encounter difficulties
    doing data analysis after data collection.

  \subsection{Workshop Surveys}

    Workshop effectiveness was assessed using a series of pre-workshop, post-workshop, and long-term workshop surveys.
    Some of the questions in the surveys were paired to perform a longitudinal analysis.

  \subsubsection{Pre-Workshop Survey}

      The pre-workshop survey had 6 main sections:
      (1) demographics,
      (2) persona,
      (3) Prior and background knowledge
      (4) Workshop Framing and Motivation

      The ``Persona'' questions presented participants with 4 learner personas:
      (1) Alex Academic,
      (2) Clare Clinician,
      (3) Patricia Programmer, and
      (4) Samir Student
      These personas were created using only data from the first of two waves of survey responses from the learner persona survey.
      Participants were asked to pick which one of the personas most resonated with them.
      %The initial 4-persona text can be found in Appendix ****. % TODO put in all 4 personas somewhere
      Subsequent persona analysis clustered into 3 groups, where Patricia Programmer was removed.

      The ``prior and background knowledge'' questions were taken from
      a 3-factor model from the first wave of the persona survey. % TODO put in original 3 factor model results into this paper? or previous?
      These questions were the highest loaded items in each factor.
      Later results from the persona factor analysis had a different set of item loadings,
      but the new items were still captured in the ``Workshop Framing and Motivation'' questions.

  \subsubsection{Post-Workshop Survey}

      The post-workshop survey had 6 main sections:
      (1) demographics,
      (2) workshop environment,
      (3) workshop framing and motivation,
      (4) summative assessment,
      (5) workshop content
      (6) open feedback.

      Questions around workshop environment, content, and open feedback
      were not used to assess workshop efficiency,
      rather, they served as feedback to the instructor for things they may need to change while teaching.
      The full survey can be found in Appendix \ref{ase:long-term-survey}.

      \paragraph{Workshop Framing and Motivation}

          There was 2 Likert tables of questions for workshop framing and motivation.
          Both sets of questions were on a 7-point Likert scale from
          ``Strongly Disagree'' to ``Strongly Agree'' and
          included a ``Neither Agree nor Disagree'' neutral term in the center.

          The first set of questions asked respondents on their level agreement with the following 7 statements:
          (1) I believe having access to the original, raw data is important to be able to repeat analysis,
          (2) I can write a small program, script, or macro to address a problem in my own work,
          (3) I know how to search for answers to my technical questions online,
          (4) While working on a programming project, if I got stuck, I can find ways of overcoming the problem,
          (5) I am confident in my ability to make use of programming software to work with data,
          (6) Using a programming language (like R or Python) can make my analyses easier to reproduce, and
          (7) Using a programming language (like R or Python) can make me more efficient at working with data.

          The second set of questions used the same 7-point scale,
          and asked respondents to rate their agreement about their ability to perform the following tasks:
          (1) Name the features of a tidy/clean dataset,
          (2) Transform data for analysis,
          (3) Identify when spreadsheets are useful,
          (4) Assess when a task should not be done in spreadsheet software,
          (5) Break down data processing into smaller individual (and more manageable) steps,
          (6) Construct a plot and table for exploratory data analysis,
          (7) Build a data processing pipeline that can be used in multiple programs, and
          (8) Calculate, interpret, and communicate an appropriate statistical analysis of the data

          These 2 sets of likert scale questions were asked in both the pre-workshop and post-workshop survey.
          The second set of questions asked each participant on how they would self-rate their ability
          to meet each of the learning objectives the learning materials sought after.
          These self-reported confidence reports served as a proxy for learning and meeting learning objectives.

      \paragraph{Summative Assessment}

          The post-workshop survey also had a summative assessment question
          where an ``untidy'' dataset is presented along with a ``tidy'' version of the same dataset
          and asked participants if they were able to
          (1) load the ``untidy'' dataset into R/Python,
          (2) Filter the data for a particular set of observations,
          (3) Save the filtered dataset to send to a colleague,
          (4) Tidy the dataset into the format presented,
          (5) Plot a histogram of one of the variables, and
          (6) Fit a logistic regression model with the dataset.
          Each of the tasks were presented as a Likert scale question,
          and respondents were asked to rate their ability to accomplish each task:
          (1) I wouldn't know where to start,
          (2) I could struggle through, but not confident I could do it,
          (3) I could struggle through by trial and error with a lot of web searchers, and
          (4) I could do it quickly with little or no use of external help.

  \subsubsection{Long-term Workshop Survey}



  \subsection{Workshop Survey Analysis}

    Participants provided a unique identifier that would be used to track their results across the 3 longitudinal surveys.
    Since participants were not forced to take any of the surveys,
    2 sets of analyses were performed:
    (1) just looking at survey responses between paired responses, and
    (2) dropped the paired survey participants and only compared unpaired survey responses.
    Composite scores for each type of analysis were created by summing up survey Likert scale responses;
    None of the survey questions analysed were reverse-coded.

  \subsubsection{Paired Responses}

      A mixed model was performed for paired results.

  \subsubsection{Un-Paired Responses}

      The non-parametric alternative to the t-test, Wilcoxon Rank Sum Test,
      was performed on the unpaired survey responses between pre-workshop and post-workshop results.

\end{document}
