\documentclass[../main.tex]{subfiles}

\begin{document}

  % sample text for abbreviations:
  %NLP is a field of computer science, artificial intelligence, and linguistics concerned with the interactions between computers and human (natural) languages.
  %
  %\nomenclature{NLP}{Natural Language Processing}
  %
  %$\sigma$ is the eighteenth letter of the Greek alphabet, and carries the 's' sound. In the system of Greek numerals, it has a value of 200.
  %
  %\nomenclature{$\sigma$}{The total mass of angels per unit area}

  The AMA is a United Staties nationally recognized association that convenes state, speciality medical socites, and critical stakeholders
  to promote the art and science of medicince and the betterment of public health
  \cite{americanmedicalassociationAmericanMedicalAssociation}.

  \nomenclature{AMA}{American Medical Association}


  The ANA is a United States organization representing the interest of the nation's registered nurses with a
  goal of improving health care quality for all
  \cite{americannursesassociationANAEnterpriseAmerican}

  \nomenclature{ANA}{American Nursing Association}


  The BLS is a US government agency that serves as the principal
  fact-finding agency for the Federal Government in labor economics and statistics
  \cite{u.s.bureauoflaborstatisticsBureauLaborStatistics}.

  \nomenclature{BLS}{Bureau of Labor Statistics}


  The COPSS is a prestigious Committee that comprises of the
  presidents, past presidents, and presidents-elect
  of several statistics and mathematics societies that are responsible for granting several awards,
  mainly the COPSS Presidents' Award for
  ``an outstanding contribution to the profession of statistics'',
  which is compared to the ``Nobel Prize of Statistics''.

  \nomenclature{COPSS}{Committee of Presidents of Statistical Societies}

  CRAN is a network of servers around the world that synchronises and stores versions of code and documentation
  for the R programming langauge and its extension libraries
  \cite{thecomprehensiverarchivenetworkComprehensiveArchiveNetwork}.

  \nomenclature{CRAN}{Comprehensive R Archive Network}


  The CSTA is a community of K-12 computer science teachers focused on supporting teachers.
  \cite{csta}.

  \nomenclature{CSTA}{Computer Science Teachers Assoc}


  CSV files are plain text datasets where each row is a line in the file,
  and column values are separated by commas (,).
  This is a specific form of a delimited file, where the comma is used as the delimiter.
  Other delimiters, such as a tab character, for tab-separated value files are also common.

  \nomenclature{CSV}{Comma Separated Value}


  FAIR is a set of guiding principles for reusability of scholarly data with an emphasis on finding and using data
  \cite{wilkinsonFAIRGuidingPrinciples2016}.

  \nomenclature{FAIR}{Findability, Accessibility, Interoperability, and Reuse of digital assets}


  A DSL is a subsection of a programming that is focused on one primary set of functions (i.e., domain).

  \nomenclature{DSL}{Domain Specific Language}


  The GAISE 2016 report describes a set recommendations to focus on
  what to teach in introductory statistics courses and
  how to teach the courses
  \cite{gaise2016}.

  \nomenclature{GAISE}{Guidelines for Assessment and Instruction in Statistics Education}


  The IDE is a software tool that makes working with a particular programming language easier to use.
  It provides an environment where writing and developing code is integrated with useful
  graphical cools.

  \nomenclature{IDE}{Integrated Development Enviornment}


  The IRB is an administrative body that reviews research proposals that involve human subjects to protect the rights and welfare of
  research participants.

  \nomenclature{IRB}{Institutional Review Board}


  KAs are topics that are a collection of topics that are related to other sub-domains.

  \nomenclature{KA}{Knowledge Area}


  An LO is a short and measureable statement of what a learner will be able to do at the end of a lesson or instructional period.
  Learning Objectives (plural) are abbreviated as LOs.

  \nomenclature{LO}{Learning Objective}


  MOOCs are online corses that do not have a limitation to the class size and typically have its learning materials freely availiable via open access.

  \nomenclature{MOOC}{Massive Open Online Courses}


  The NIH is a government agency in the United States primarily responsible for biomedical and public health research.

  \nomenclature{NIH}{National Institutes of Health}


  The NNLM is part of the United States Department of Health and Human Services
  with the goal of improving access to biomedical information to U.S. health professionals and the public
  \cite{nationallibraryofmedicineUsNNLM}.

  \nomenclature{NNLM}{National Network of Libraries of Medicine}


  OHDSI (pronounced ``Odyssey'') is a multi-stakeholder, interdisciplinary, open-science collaborative
  to bring out the value of health data through large-scale analytics
  \cite{observationalhealthdatasciencesandinformaticsOHDSIObservationalHealth}.

  \nomenclature{OHDSI}{Obserbational Health Data Sciences and Informatics}


  OMOP is a common data model that allows for the systematic analysis of disparate observational databases
  \cite{observationalhealthdatasciencesandinformaticsohdsiobservationalhealthOMOPCommonData}.

  \nomenclature{OMOP}{Observational Medical Outcomes Partnership}


  OSEMN (pronounced ``awesome'') is the acronym in Hiliary Mason and Chris Wiggins's Snice taxonomy that defines
  the rough order of the data science process:
  obtain, scrub, explore, model, and i(n)terpret
  \cite{masonTaxonomyDataScience2010}.

  \nomenclature{OSEMN}{Obtain, Scrub, Explore, Model, and i(N)terpret}


  PBCs is a for-profit company designation that can afford legal protection
  to prioritize company values over shareholder returns.

  \nomenclature{PBC}{Public Benefit Corporation}


  Pedagogical content knowledge is the subject matter knowledge for the act of teaching.
  It includes knowing what are the most commonly taught topics in an area,
  and knowing what makes specific topics difficult to learn
  \cite{shulmanThoseWhoUnderstand1986}.

  \nomenclature{PCK}{Pedagogical Content Knowledge}


  The REPL is used in interactive programming sessions where users are able to
  submit code (read) to be interpreted (evaluate) and the results are returned (print).
  This process then starts over where the programming language waits for the next command (loop).
  It allows the user to program interactively (as opposed to compiling the code first).

  \nomenclature{REPL}{Read-Evaluate-Print-Loop}

  UCD puts effort into thinking about the end user's needs above the needs of the creator in a self-centered design
  \cite{pruittPersonaLifecycleKeeping2006, tognazziniTogSoftwareDesign1748}.

  \nomenclature{UCD}{User-Centered Design}

\end{document}
