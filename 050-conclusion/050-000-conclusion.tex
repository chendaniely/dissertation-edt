\documentclass[../main.tex]{subfiles}
\begin{document}

The work in this dissertation started out exploring how to create better data science educational materials
by creating and using learning personas.
The learning materials provided the ability to self-study,
but the research component focused on the in-class (virtual or in-person) learning.
We discovered that the workshop was beneficial when looking at self-reported confidence of competing learning objective tasks.
Having more instruction and guidance can help with getting over the activation energy to get started and learn these skills
and the learning personas and relevant teaching examples can help with internal motivation
to get over the initial learning curve.

However, we also discovered that there was a drop in confidence of competing learning objective tasks
in the long-term study (at least 4 months later).
This has lead us to conclude that more efforts should be spend on long-term learning,
rather than creating more training materials in an already crowded market.
For our audience of interest, working professionals in the medical field,
this points to a problem costs to attending workshops and classes are wasted.
These costs can be monetary (i.e., paying for the instruction), but there is also a time cost.
The decrease in long-term confidence suggests that the value of these data sciences courses are short-term.

One hypothesis is from the lack of using and practicing the skills from the workshop.
This plays into Malcolm Gladwell's ``1,000 hour rule'' and Ericsson's notion of ``deliberate practice''
\cite{gladwellOutliersStorySuccess2011, ericssonDeliberatePracticeProposed2019, ericssonExpertExceptionalPerformance1996}.
Instead of simply providing datasets for examples,
the personas we identified can be used to create and curate examples.
This can help with deliberate and focused practice to actually maintain and build on knowledge and skills.

Providing datasets to work on data skills is one mechanism where learners can practice skills.
However, these datasets need to be curated as learners may not always know where to find them let alone loading them for practice.
The OpenX community can help with the curation of datasets,
and communities like TidyTuesday publish weekly datasets where participants can explore datasets on their own.
The personas we identified in this dissertation work can be used to refine these public datasets
by providing different levels of questions to explore in a dataset.
Combining focused case-study examples and the community gives a way for more experienced learners to mentor newer learners,
and spreading out case-study examples can serve as a model for spaced repetition to practice foundational skill,
while motivating learning a new skill.

While the work in this dissertation did create its own introductory workshop materials,
it is possible that the effort in curating the learning materials could have been better served creating exercise questions
to be used during and after the actual workshop.
It may be possible that the materials used to teach need not be as domain focused
to motibate learners, if more releveant examples come in the form of exercise questions and long-term case-study examples.
New materials can link to existing materials and create a learning path,
rather than re-creating the same materials.
This leverages the plethora of existing data science materials,
and puts a long-term focus for specific domains by working on domain-specific exercises.
Focusing the domain-specific learning materials on exercies, instead of the introductory text,
can help with the resusability paradox where the potential for reuse clashes is inversly related to pedagogical value
because the specific context needed for pedagocial value, but reduces its ability to transfer to other contexts
\cite{wileyReusabilityParadox2002}.

What may be more important for new learners is focusing on long-term learning and practice.
This can be done with exercise questions and case-studies.
Having a centralized location where datasets and exercise questions tagged with what skills are being practiced,
extend the current work with the TidyTuesday project,
and can be more beneficial to educators and learners without having to recreate yet another
introductory text.
These suggestions are mainly focused on the population of working adults.
K-12, university, and higher education programs typically expose their students
over the degree program to new topics and knowledge to build skills.

Communities of practice are one mechanism to help balance the amount of resources needed to teach.
Auto-graders can help alleviate resources to check example solutions.
Learning materials can be published in an open source platform,
and communities can work on the long-term maintenance of these materials.
These communities can also be centralized hubs where exercises can be posted.
If these exercises were tagged with skill and domain information,
educators can better filter teaching exercise for learners,
and learners can explore examples on their own.

One of the main goals as educators is to inspire the next generation.
Having compassion not only impacts those we interact with, but also beneficial to ourselves
\cite{trzeciakCompassionomics2019}.
Greg Wilson lists ``The Rules'' for teaching \cite{wilson2019teaching}, and it begins with: ``Be kind: all else are details''.

\end{document}
